%--------------------------------------------------------------------%
%
% Berkas utama templat LaTeX.
%
% author Petra Barus, Peb Ruswono Aryan, Faris Rizki Ekananda
%
%--------------------------------------------------------------------%
%
% Berkas ini berisi struktur utama dokumen LaTeX yang akan dibuat.
%
%--------------------------------------------------------------------%

\documentclass[bahasa, 12pt, a4paper, onecolumn, oneside, final]{report}

\input{config/if-itb-thesis.sty}

\makeatletter

\makeatother

\addbibresource{references.bib}

\begin{document}

%Basic configuration
\title{Pengembangan Akselerator Perangkat Keras Berbasis RISC-V untuk \textit{Reinforcement Learning}}
\date{}
\author{
	Muhammad Sulthan Mazaya\ \ \ \ NIM: 13320028
}
\newcommand\tanggalpengesahan{17 Mei 2024}

\pagenumbering{roman}
\setcounter{page}{1}

% setting caption
\captionsetup{labelsep=space}
\captionsetup{labelfont=bf}
\setlength{\belowcaptionskip}{0pt}

\clearpage
\pagestyle{empty}

\begin{center}
    \smallskip
    
    \large \MakeUppercase{\bfseries \thetitle}
    
    \vfill
    
    \large \MakeUppercase{\bfseries Laporan Kemajuan Tugas Akhir}
    
    \vfill
    
    \begin{figure}[h]
        \centering
        \includegraphics[width=0.15\textwidth]{cover-ganesha.jpg}
    \end{figure}
    
    \vfill
    
    

    \small{
    Oleh

    \theauthor
    }

    \vfill

    \large
    \uppercase{
        \bfseries
        Program Studi Teknik Fisika \\
        Fakultas Teknologi Industri \\
        Institut Teknologi Bandung
    }
    
    \bfseries{2024}

    \vfill
    
\end{center}

\clearpage

\clearpage
\chapter*{ABSTRAK}
\addcontentsline{toc}{chapter}{ABSTRAK}

\begin{center}
	\begin{singlespace}
		\large\bfseries\MakeUppercase{\thetitle}
		
		\normalfont\normalsize
		
		Oleh\\
		\bfseries{Muhammad Sulthan Mazaya \hspace{5mm} NIM: 13320028}
		
		\vspace{5mm}
		\large\bfseries{(Program Studi Teknik Fisika)}
		\vspace{5mm}
		
	\end{singlespace}
\end{center}

\begin{singlespace}
	\small
	\textit{Reinforcement Learning} (RL) merupakan salah satu kerangka pengembangan agen
	otonom yang marak digunakan. RL merupakan sebuah alternatif solusi pemodelan
	untuk sebuah permasalahan pada sebuah sistem yang terlalu kompleks untuk dibuat
	model secara matematis maupun algoritmik. Dengan demikian, RL marak
	digunakan pada permasalahan seperti robotika dan agen pengemudi otonom.
	Namun, meskipun sebuah alternatif yang tepat untuk banyak permasalahan, sering
	kali RL sulit untuk digunakan karena keterbatasan sumber daya perangkat keras.
	Hal ini disebabkan karena umumnya RL memerlukan sumber daya perangkat keras
	yang signifikan. Hal ini dapat menjadi hambatan saat ingin melakukan \textit{deployment
		model} RL pada komputer dengan sumber daya terbatas, seperti perangkat \textit{Internet
		of Things} (IoT) atau \textit{edge computing}. Oleh karena itu, pada penelitian ini dilakukan
	perancangan sebuah desain akselerator perangkat keras yang mampu membuat
	daya komputasi yang diperlukan untuk komputasi algoritma RL berkurang. Desain
	akselerator perangkat keras ini akan dilakukan pada sebuah \textit{Field Programmable
		Gate Array} dengan arsitektur RISC-V, sebuah arsitektur instruksi yang boleh
	digunakan secara terbuka, dalam bentuk \textit{co-processor}. Hasil yang telah dicapai
	pada tugas akhir ini berupa konfigurasi perangkat keras dan perangkat lunak,
	beserta desain dari perangkat lunak dan perangkat keras yang akan
	diimplementasikan.
	
	Kata kunci: \textit{reinforcement learning}, \textit{co-processor}, \textit{field programmable gate array}, RISC-V
\end{singlespace}
\clearpage

\clearpage
\chapter*{\textit{ABSTRACT}}
\addcontentsline{toc}{chapter}{Abstract}

\vspace{5mm}

\begin{center}
	\center
	\large\bfseries\MakeUppercase{\textit{DEVELOPMENT OF RISC-V BASED HARDWARE ACCELERATOR FOR REINFORCEMENT LEARNING}}
	
	\normalfont\normalsize
	
	By\\
	\bfseries{\textit{Muhammad Sulthan Mazaya \hspace{5mm} NIM: 13320028}}
	
	\vspace{5mm}
	\large\bfseries{\textit{(Engineering Physics Study Program)}}
	\vspace{5mm}
\end{center}


\begin{singlespace}
	\small
	\textit{Reinforcement Learning (RL) is one of the popular frameworks for developing
		autonomous agents. RL serves as an alternative modeling solution for problems in
		systems that are too complex to be mathematically or algorithmically modeled. As
		a result, RL is widely employed in domains such as robotics and autonomous driver
		agents. However, despite being a suitable alternative for many problems, RL is
		often challenging to use due to hardware resource limitations. This is primarily
		because RL typically demands significant hardware resources. This can be a
		hindrance when deploying RL models on computers with limited resources, such as
		Internet of Things (IoT) devices or edge computing platforms. Therefore, in this
		research, a akselerator perangkat keras design is proposed to reduce the
		computational power required for RL algorithm computation. This akselerator
		perangkat keras design will be implemented on a Field Programmable Gate Array
		with an RISC-V, an open instruction set architecture, architecture in the form of a
		co-processor. The results achieved in this final project consist of hardware and
		software configurations, along with the design of the software and hardware that
		will be implemented.
	}
	
	\textit{Keywords: reinforcement learning, co-processor, field programmable gate array, RISC-V}
\end{singlespace}
\clearpage

\clearpage

\clearpage
\chapter*{} % supaya page number muncul
\addcontentsline{toc}{chapter}{Halaman Pengesahan}

\begin{center}
    \begin{singlespace}
        \large \bfseries \MakeUppercase{\thetitle}

        \large \MakeUppercase{Halaman Pengesahan}

        \vspace{15mm}
        
        \normalsize \normalfont 
        Oleh

        \bfseries
        Muhammad Sulthan Mazaya \hspace{5mm} NIM: 13320028
        
        (Program Studi Teknik Fisika) \\

        \vspace{10mm}

        \normalsize \normalfont 
        Institut Teknologi Bandung \\

        \vspace{20mm}

        Menyetujui\\
        Tim Pembimbing

        \vspace{10mm}

        Tanggal \tanggalpengesahan
        
        \vspace{0.5cm}

        \begin{multicols}{2}
            Pembimbing 1

            \vspace{25mm}

            Dr. Ir. Eko Mursito Budi, M.T.\\
            NIP. 196710061997021001
            
            \columnbreak


            Pembimbing 2

            \vspace{25mm}

            Dr. Eng. Infall Syafalni, S.T., M.Sc.\\
            NIP. 198707072019031028
        \end{multicols}
    \end{singlespace}
\end{center}
\clearpage


\pagestyle{plain}

\chapter*{Kata Pengantar}
\addcontentsline{toc}{chapter}{Kata Pengantar}

Puji syukur kepada Tuhan Yang Maha Esa karena atas berkat-Nya, telah diselesaikan tugas akhir dengan judul "\thetitle".
Atas kontribusi baik secara langsung maupun tidak langsung, penulis ingin
mengucapkan terima kasih kepada:

\begin{enumerate}
	\item Bapak Dr. Ir. Eko Mursito Budi, M.T. dan Bapak Dr. Eng. Infall Syafalni, S.T., M.Sc. selaku dosen pembimbing tugas akhir yang telah membimbing, memberikan arahan, dan mendukung penulisan laporan kemajuan tugas akhir ini.
	\item Bapak Nana Sutisna, S.T., M.T., Ph.D. dan Kak Yahwista Salomo, S.T. yang telah memberikan bimbingan dan bantuan arahan mengenai metodologi penelitian yang akan dilakukan dan dicantumpan pada laporan kemajuan tugas akhir.
	\item Orang tua dan keluarga Penulis yang selalu memberikan dukungan penuh dan menyediakan kebutuan serta doa sehingga Penulis dapat menyelesaikan penulisan laporan kemajuan tugas akhir.
	\item Seluruh Dosen dan Staff Program Studi Teknik Fisika Institut Teknologi Bandung yang telah memberikan banyak bantuan agar dapat tertulisnya laporan kemajuan tugas akhir.
	\item Prudensia Fairuz Zhafirah yang telah memberikan dukungan moral dan menemani penulisan laporan kemajuan tugas akhir ini.
	\item Teman-teman Teknik Fisika 2020 yang telah memberikan dukungan moral untuk menyelesaikan penulisan laporan kemajuan tugas akhir ini.
	\item Semua pihak yang tidak dapat disebutkan satu per satu. Semoga laporan kemajuan tugas akhir ini dapat memberikan gambaran baik tentang tugas akhir yang akan dilakukan.
\end{enumerate}

Akhir kata, penulis mengucapkan terima kasih kepada semua pihak yang telah terlibat dalam pengerjaan tugas akhir ini. Penulis juga ingin menyampaikan mohon maaf apabila terdapat kesalahan maupun kekurangan dalam laporan tugas akhir ini. Penulis berharap semoga tugas akhir ini dapat bermanfaat bagi pembaca dan riset-riset kedepannya.

\begin{flushright}
	\vspace{0.5cm}
	Bandung, \tanggalpengesahan
	
	
	\vspace{1.5cm}
	
	Muhammad Sulthan Mazaya
\end{flushright}


\titleformat*{\section}{\centering\bfseries\Large\MakeUpperCase}
\titlespacing*{\chapter}{0pt}{0pt}{4pt}

% Setting judul toc, lot, lof, bib
% \renewcommand{\cftXpagefont}{\normalfont} % make font non bold
\renewcommand{\contentsname}{DAFTAR ISI}
\renewcommand{\listfigurename}{DAFTAR GAMBAR}
\renewcommand{\listtablename}{DAFTAR TABEL}
\renewcommand{\bibname}{DAFTAR PUSTAKA}

\renewcommand{\figurename}{Gambar}
\renewcommand{\tablename}{Tabel}

\tableofcontents
% \listofappendices daftar lampiran
\listoffigures
\listoftables
\clearpage
\chapter*{DAFTAR SINGKATAN DAN LAMBANG}
\addcontentsline{toc}{chapter}{DAFTAR SINGKATAN DAN LAMBANG}

\begin{singlespace}
	\begin{acronym}
		\acro{RL}{\textit{Reinforcement Learning}}
		\acro{EC}{\textit{Edge Computing}}
		\acro{IoT}{\textit{Internet of Things}}
		\acro{FPGA}{\textit{Field Programmable Gate Array}}
	\end{acronym}
\end{singlespace}

\clearpage


\newpage

\titleformat*{\section}{\bfseries\normalsize}
\titleformat*{\subsection}{\bfseries\normalsize}
\titlespacing*{\chapter}{0pt}{0pt}{*1}
\titlespacing*{\section}{0pt}{10pt}{*1}
\pagenumbering{arabic}

%----------------------------------------------------------------%
% Konfigurasi Bab
%----------------------------------------------------------------%
\setcounter{page}{1}
\renewcommand{\chaptername}{BAB}
%----------------------------------------------------------------%

%----------------------------------------------------------------%
% Dafter Bab
% Untuk menambahkan daftar bab, buat berkas bab misalnya `chapter-6` di direktori `chapters`, dan masukkan ke sini.
%----------------------------------------------------------------%
\chapter{Pendahuluan}

Konten pada bab ini berisi terkait gambaran umum dan permasalahan yang akan diselesaikan dalam tugas akhir ini. Bab ini akan dimulai dari penjelasan latar belakang dari masalah yang diselesaikan, rumusan masalah, tujuan, batasan masalah, metodologi yang digunakan, dan berakhir pada sistematika penulisan tugas akhir ini.

\section{Latar Belakang}

% Outline
% - Information Retrieval Application, Apache Lumen, Elastic Search
% - Java
% - Resource Limit
% - Kubernetes, Pods, PersistentVolume, StatefulSet, Horizontal & Vertical Autoscaler
% - Metrics
% - Kontrol Adaptif

\textit{Information Retrieval} (IR) adalah penemuan bahan seperti dokumen yang bersifat terstruktur yang memenuhi kebutuhan informasi dari dalam koleksi besar yang tersimpan di dalam komputer \parencite{introtoinforetri}. IR saat ini sangat sering dilakukan contohnya dalam pencarian informasi berkaitan dengan representasi, penyimpanan, pengaturan, dokumen, halaman web, katalog online, catatan, dan objek multimedia. Salah satu aplikasi yang membantu dalam melakukan hal tersebut adalah Elastic Search.

\textit{Elasticsearch} merupakan mesin pencarian dan analitik terdistribusi yang dibangun di Apache Lucene. \textit{Elasticsearch} telah dengan cepat menjadi mesin pencari paling populer dan biasa digunakan untuk analisis log, pencarian teks lengkap, inteligensi keamanan, analisis bisnis, dan kasus penggunaan inteligensi operasional \parencite{elasticsearch}.

Proses \textit{Elastic Search} menggunakan JVM. Umumnya pada \textit{Elastic Search}, hampir 50 persen memori yang tersedia akan dialokasikan ke JVM. Pemrosesan data berukuran besar dalam \textit{in-memory computing} dalam JVM memakan sangat banyak memory \parencite{jvm}. Dalam konteks \textit{Elastic Search}, mesin JVM memerlukan memakai memori karena Apache Lucene membutuhkan melakukan pengindeksan. Sedangkan, 50 persen sisanya akan dipakai untuk melakukan \textit{caching} dalam memori agar mempercepat pencarian file yang sering diakses.

Hal ini menyebabkan \textit{Elastic Search} akan memakan memori sebanyak-banyaknya untuk dipakai \textit{caching}. Sedangkan, dalam segi kebutuhan, belum tentu semua data pada \textit{cache} akan dipakai. Ada kalanya waktu saat banyak terjadi \textit{miss} karena data yang dicari terlalu bervariasi. Atau kondisi lain seperti \textit{cost cache} yang terlalu besar dan tidak sebanding dengan pengunaan memori pada suatu \textit{Kubernetes Cluster}. Kondisi seperti itu tidak menjadi masalah ketika sebuah \textit{node} sedang kosong dan punya banyak memori. Namun, akan menjadi masalah ketika ada aplikasi lain yang ingin memakai bersama \textit{node} tersebut dan memori telah dipakai habis oleh \textit{Elastic Search}.

Untuk menangani hal tersebut, Kubernetes sendiri sudah memiliki fitur bernama \textit{resource limit}. Fitur ini digunakan untuk membatasi pengunaan sumber daya oleh sebuah aplikasi. Namun, \textit{resource limit} ini bersifat statik dan tidak adaptif. Jika ingin mengubahnya, diperlukan pengubahan konfigurasi melalui file \textit{deployment} atau perintah \textit{kubectl}. Sedangkan, performa suatu \textit{information retrieval} sangat dinamis dan apabila ingin mengontrol pengunaan memori berdasarkan performa membutuhkan alat yang sangat dinamis.

Kubernetes sendiri sudah memiliki \textit{auto-scaler} yang dinamis. Pada \textit{auto-scaler} versi horizontal, komponen ini akan mereplikasi secara otomatis ketika performa memburuk. Namun, \textit{auto-scaler} ini akan mematikan dan menyalakan \textit{node} baru pada kluster \textit{Elastic Search}. Dan oleh karena itu, akan terjadi banyak \textit{balancing} dan replikasi \textit{shard} pada \textit{Elastic Search} yang menyebabkan banyak \textit{overhead} apabila terlalu sering terjadi penambahan atau pengurangan \textit{node} pada kluster \textit{Elastic Search}. Sedangkan, pada \textit{auto-scaler} versi vertikal, komponen ini masih dikembangkan oleh Kubernetes, dan mengharuskan Kubernetes untuk melakukan \textit{restart} pada \textit{node} yang di-\textit{scale} secara vertikal yang berarti kasusnya tidak jauh beda dengan \textit{scaling} secara horizontal.

\section{Rumusan Masalah}

Berdasarkan latar belakang yang ada, rumusan tugas akhir ini adalah sebagai berikut.
\begin{enumerate}
    \item Bagimana cara untuk mengontrol pengunaan memori \textit{Elastic Search} tanpa melakukan \textit{restart} atau penambahan \textit{node} baru?
    \item Bagaimana dampak memori yang dipakai terhadap kinerja aplikasi?
    \item Bagaimana cara untuk mengontrol alokasi sumber daya suatu aplikasi tanpa mengurangi kinerja aplikasi?
    \item Bagaimana dampak \textit{adaptive control} terhadap efisiensi penggunaan sumber daya?
\end{enumerate}

\section{Tujuan}

Tujuan yang akan dicapai untuk tugas akhir ini adalah sebagai berikut.

\begin{enumerate}
    \item Mengembangkan sebuah komponen yang menarik \textit{metrics} dari \textit{Elastic Search}.

    \item Mengembangkan sebuah komponen yang mempelajari dampak sumber daya berupa memori dan kondisi terhadap \textit{metrics} sehingga dapat memprediksi ukuran memori yang cocok dan efisien.

    \item Mengetahui dampak pengaplikasian \textit{adaptive control} tersebut terhadap efisiensi penggunaan memori pada \textit{Elastic Search}.
\end{enumerate}

\section{Batasan Masalah}

Terdapat batasan yang diambil dalam pelaksanaan tugas akhir ini, yaitu sebagai berikut.

\begin{enumerate}
    \item Solusi yang diimplementasikan akan pada level \textit{Pods} Kubernetes.
    \item Solusi yang diimplementasikan hanya akan spesifik pada \textit{Elastic Search}.
    \item Solusi yang diajukan tidak akan memakai replika dan akan dianggap sebagai solusi untuk satu buah \textit{node Elastic Search}.
    \item Agar mensimulasikan lingkungan dengan sumber daya terbatas, solusi yang diajukan akan memakai lingkungan Kubernetes dengan \textit{single node}.
 \end{enumerate}

\section{Metodologi}

Terdapat metodologi yang digunakan untuk melaksanakan tugas akhir ini, berikut adalah tahapan pelaksanaan.
\subsection{Identifikasi Permasalahan}
Tahapan ini adalah tahapan untuk melakukan identifikasi permasalahan. Hasil dari tahapan ini dijadikan gagasan utama dan arah kerja dalam tugas akhir ini.
\subsection{Perancangan Solusi}
Setelah mengidentifikasi permasalahan, dilakukan perancangan solusi yang bertujuan untuk mencari metode dan pendekatan yang dapat dikembangkan untuk menyelesaikan permasalahan yang ada. Analisis ini dimulai dari eksplorasi metode melalui studi literatur.
\subsection{Implementasi}
Setelah merancang solusi, gagasan tersebut akan dikembangkan dan diimplementasikan. Tahap ini akan menghasilkan dua hal sebagai berikut.
\subsubsection{Komponen \textit{Metrics Controller}}
Komponen yang akan mencatat kinerja dari aplikasi terhadap variabel sumber daya yang dipakai.
\subsubsection{Komponen \textit{Memory Controller}}
Komponen yang akan memutuskan alokasi sumber daya berdasarkan catatan kinerja yang telah disimpan menggunakan pembelajaran mesin.
\subsection{Eksperimen Pengujian}
Hasil implementasi yang sudah dibuat pada tahapan sebelumnya akan diuji pada tahapan ini. Tahapan ini juga akan melakukan analisis dampak pengaplikasian \textit{adaptive control} terhadap efisiensi sumber daya.
\subsection{Evaluasi Hasil Eksperimen}
Hasil eksperimen pada tahapan sebelumnya akan dianalisis dan dievaluasi. Jika kurang memuaskan, hasil dari eksperimen ini bisa digunakan untuk meningkatkan dampak \textit{adaptive control} terhadap efisiensi sumber daya yang dimiliki. Namun, jika sudah sesuai yang diekspektasikan, maka hasil eksperimen ini akan cukup untuk membuktikan efisiensi yang tercipta dari implementasi \textit{adaptive control}.

\section{Sistematika Pembahasan}

Konten dari Tugas Akhir ini akan dibagi menjadi lima bab sebagai berikut.
\begin{enumerate}
    \item Pendahuluan
    \item Studi Literatur
    \item Analisis Masalah dan Rancangan Solusi
    \item Implementasi dan Pengujian
    \item Kesimpulan dan Saran
\end{enumerate}

Pada Bab I akan dijelaskan gagasan utama dari tugas akhir ini yang berisi dari latar belakang, rumusan masalah, tujuan, batasan, metodologi hingga sistematika pembahasan.

Selanjutnya, Bab II akan menjelaskan hasil studi literatur yang berkaitan dengan pengerjaan tugas akhir ini. Bab II ini berisi tentang pemahaman dasar seputar topik yang akan dibahas pada tugas akhir ini.

Pada Bab III akan dijelaskan ulang masalah serta latar belakang untuk menyusun rancangan solusi. Di bab ini juga akan dipaparkan beberapa rancangan solusi yang kemudian akan dijelaskan lebih lanjut dan dipilih sebagai topik yang akan diimplementasikan pada bab selanjutnya.

Bab IV ...

Bab V akan menjadi penutup pada tugas akhir ini. Konten pada bab ini akan menjelaskan jawaban terhadap rumusan masalah pada bab I. Pada bab ini juga akan disebutkan saran-saran perbaikan yang bisa dipakai untuk penelitian berikutnya. Bab ini akan menyimpulkan hasil implementasi dan rancangan solusi terhadap masalah yang sudah diidentifikasi.
\chapter{Studi Literatur}

Pada bab ini, akan diisi oleh studi literatur, hal-hal yang berkaitan dengan topik persoalan tugas akhir akan dipaparkan dalam bab ini guna untuk memberikan informasi mengenai dasar teori dan studi yang dipakai. Bab ini diharapkan membantu pembaca untuk mengerti dalam membaca penelitian tugas akhir ini.

\section{\textit{Information Retrieval}}

\section{Apache Lucene}

\section{Elastic Search}

\section{Java \textit{Virtual Machine}}

\section{\textit{Indexing}}

\section{\textit{Caching}}

\section{Kubernetes}

\subsection{Pods}

\subsection{\textit{Auto Scaler}}

\subsubsection{\textit{Horizontal Auto Scaler}}

\subsubsection{\textit{Vertical Auto Scaler}}

\subsection{\textit{Deployment File and Resource Limit}}

\subsection{Kubernetes Client}

\section{\textit{Metrics}}

\section{Pembelajaran Mesin}

\subsection{\textit{Supervised Learning}}

\subsection{\textit{Support Vector Machine}}
% \blankpage
\chapter{Analisis Persoalan dan Rancangan Solusi}

Tujuan utama penulisan bab ini adalah untuk menguraikan rencana penyelesaian masalah tugas akhir yang akan dieksekusi secara utuh pada saat pelaksanaan Tugas Akhir II. Bab ini merupakan bab penutup Laporan Tugas Akhir I yang dapat dipandang sebagai bab yang akan menjembatani perpindahan ke proses pelaksanaan Tugas Akhir II. Pengembangan lebih lanjut dari bab ini dapat menjadi bagian dari bab Deskripsi Solusi pada Laporan Tugas Akhir.

\section{Analisis Persoalan}

Basis data, \textit{information retrieval} dan server memori seringkali digunakan sebagai \textit{technology stack} aplikasi zaman sekarang.
Ketiga jenis aplikasi ini akan menggunakan memori semaksimal yang mereka dapat gunakan ketika memiliki jumlah data yang besar.
Kegunaan dari memakai memori biasanya dipakai untuk melakukan \textit{caching} yang sebenarnya pada aplikasi skala besar tidak akan terlalu signifikan karena permintaan masuk yang terlalu beragam. 
Akibatnya, memori yang ditahan oleh aplikasi tersebut tidak dapat digunakan oleh aplikasi lainnya.
Permasalahan ini lebih nyata terasa ketika aplikasi tersebut diletakkan pada sebuah \textit{pods} Kubernetes. Karena, sebuah \textit{pods} harus bisa dimuat dalam sebuah \textit{node} yang sumber dayanya terbatas dan belum tentu berukuran besar. Meskipun \textit{cloud} seringkali dianggap sumber daya tak terbatas, namun penggunaan memori yang terlalu besar dan tidak berakibat terlalu signifikan terhadap performa adalah sangat tidak efisien.

Salah satu aplikasi \textit{information retrieval} yang sering dipakai adalah \textit{Elastic Search}. 
Aplikasi ini membungkus kakas \textit{Apache Lucene} yang dipermudah dengan membuat standar antarmuka berupa \textit{HTTP Request}.
\textit{Elastic Search} juga diciptakan guna untuk menghilangkan keterbatasan bahasa, hanya aplikasi berbasis Java yang dapat menggunakan kakas \textit{Apache Lucene}.
Sayangnya, aplikasi ini sangat boros dalam hal memori dan cenderung untuk mengambil memori sebanyak yang diberikan.

\textit{Elastic Search} dijalankan diatas \textit{Java Virtual Machine} (JVM) yang pada umumnya memiliki command flag untuk membatasi memori yang akan digunakan. Tidak hanya itu, \textit{Kubernetes Pods} juga memiliki parameter berupa \textit{resource limit} yang dapat diubah-ubah untuk membatasi penggunaan CPU dan memori. Cara tersebut dapat dilakukan jika ingin diraih efisiensi sumber daya dengan mengacuhkan kinerja dari aplikasi dan kondisi jumlah permintaan dalam satu satuan waktu dalam jangka waktu tertentu. Oleh karena itu, diperlukan kontrol adaptif yang mengubah secara dinamis.

\section{Analisis Solusi}

Untuk membuat kontrol adaptif untuk menangani masalah efisiensi sumber daya berdasarkan kinerja aplikasi serta kondisi jumlah permintaan dalam satu satuan waktu dalam jangka waktu tertentu, akan diajukan dua buah pendekatan solusi. Berikut adalah analisis dari dua pendekatan yang diajukan pada tugas akhir ini.

\subsection{Vertical Pod Autoscaler dari Kubernetes}

Kubernetes sendiri saat ini sedang mengembangkan Vertical Pod Autoscaler (VPA), \parencite{vpa}. Fitur ini sudah bisa dipakai meskipun masih dalam tahap pengembangan. Namun, jika menggunakan pendekatan ini, terdapat beberapa \textit{drawback}. Pertama, VPA perlu melakukan \textit{restart} terhadap \textit{pod} yang ingin dibesarkan sedangkan \textit{Elastic Search} menggunakan sistem sharding, apabila mematikan salah satu \textit{Node Elastic Search} maka hal tersebut akan menyebabkan \textit{Elastic Search} perlu melakukan \textit{balancing shard data} setiap kali \textit{autoscale} yang tentunya akan memakan ketersediaan dan sumber daya. Kedua, VPA menggunakan \textit{metrics} yang didapatkan dari Kubernetes bukan dari Elastic Search, hal ini akan menyebabkan kurangnya akurasi dan atau tidak tercapainya tujuan untuk membebaskan memori yang dipakai namun dampaknya tidak signifikan terhadap kinerja \textit{Elastic Search} itu sendiri. Oleh karena itu, rancangan solusi dengan hal ini dirasa kurang cocok.

\subsection{Sistem Kontrol Adaptif}
\label{sec:sistemkontroladaptif}

Sistem Kontrol Adaptif akan memanfaatkan \textit{metrics} yang didapat dari \textit{Elastic Search} secara periodik. Data tersebut akan dijadikan faktor dalam membuat keputusan untuk memperbesar atau memperkecil limit memori \textit{Elastic Search} tersebut. Sistem tersebut secara umum akan dibagi menjadi dua komponen, yaitu \textit{Metrics Collector} dan \textit{Memory Controller}.

\section{Rancangan Solusi}

Seperti yang sudah dijelaskan pada bagian sebelumnya, \ref{sec:sistemkontroladaptif}, Sistem Kontrol Adaptif akan disusun atas dua komponen, yaitu sebagai berikut.
\subsection{Komponen \textit{Metrics Collector}}
\label{sec:metricscontroller}

Komponen ini bertugas untuk menarik data \textit{metrics} dari \textit{Elastic Search} menggunakan HTTP Client dan Database Connector Client. Data tersebut lalu akan disimpan ke sebuah database relasional yang dapat digunakan sebagai data historis. Sehingga, kedepannya dapat dilakukan analisis diagnostik dan analisis prediktif.

\subsection{Komponen \textit{Memory Controller}}

Komponen ini bertugas untuk membuat keputusan untuk memperbesar, membiarkan atau memperkecil limit memory \textit{Elastic Search}. Komponen ini akan menggunakan data yang dikumpulkan oleh komponen Metrics Controller, \ref{sec:metricscontroller}. Kakas yang akan digunakan oleh komponen ini adalah Kubernetes Client Library, Database Connector Client, dan Library Machine Learning, jika diperlukan.

Adapun algoritma yang akan digunakan untuk membuat keputusan. Saat ini ada beberapa pilihan algoritma yang dapat diterapkan sebagai pembuat keputusan.

\subsubsection{Greedy, Trial and Error}
\label{sec:greedytrialerror}

Secara periodik, algoritma ini akan memaksa memangkas limit memori sebesar 50 persen dari kondisi sekarang.
Apabila kinerja \textit{Elastic Search} memburuk, algoritma akan mengembalikan limit memori sebesar 50 persen dari kondisi saat itu.
Hal ini akan terus dilakukan sampai konfigurasi algoritma sistem diubah.

\subsubsection{Support Vector Machine}

Awalnya, algoritma ini akan memakai algoritma sebelumnya, \ref{sec:greedytrialerror}, untuk mengumpulkan data.
Setelah data terkumpul, algoritma ini akan memprediksi menggunakan  Support Vector Machine untuk memutuskan melakukan peningkatan atau pengurangan limit memori.
Jika terjadi perubahan trend atau kebiasaan, algoritma akan melakukan training ulang dari data-data terbaru.
Support Vector Machine ini akan menerima \textit{input} berupa memori yang dipakai, \textit{request per second}, waktu untuk melakukan pencarian, waktu untuk melakukan aggregasi, waktu untuk melakukan penggabungan.
\chapter{Implementasi dan Pengujian}
Bab ini akan menjelaskan proses implementasi dari rancangan solusi yang telah dikaji pada Bab III. Setelah pembahasan terkait implementasi, akan dilanjutkan dengan pemaparan hasil uji terkait implementasi yang telah dibuat.

\section{Lingkungan}

Sistem kontrol adaptif akan diimplementasikan di lingkungan komputer lokal. Berikut adalah lingkungan perangkat keras dan perangkat lunak secara terperinci.

\subsection{Lingkungan Perangkat Keras}
\blindtext

\subsection{Lingkungan Perangkat Lunak}
\blindtext

\section{Implementasi}

Bagian ini akan menjelaskan tentang implementasi sistem kontrol adaptif secara terperinci.

\subsection{Batasan Implementasi}
\blindtext

\subsection{Kakas yang Digunakan}
\blindtext

\subsection{Komponen \textit{Metrics Fetcher}}
\blindtext

\subsection{Komponen \textit{Predictor}}
\blindtext

\subsection{Komponen \textit{Rule Manager}}
\blindtext

\subsection{Komponen \textit{Resource Controller}}
\blindtext

\subsection{Komponen \textit{Adaptive Control}}
\blindtext

\subsection{Tampilan Implementasi}
\blindtext

\subsection{Sistem Kontrol Adaptif dengan Model Prediktif berbasis \textit{Time Series}}

Seperti yang sudah dijelaskan pada bagian rancangan solusi (\ref{sec:rancangan-solusi}), sistem kontrol adaptif akan diimplementasikan dengan beberapa komponen penyusun, diantaranya adalah sebagai berikut.

\subsubsection{Komponen \textit{Metrics Fetcher}}
\textit{Metrics Fetcher} merupakan komponen yang berbeda dibanding komponen lainnya, karena komponen ini berjalan pada \textit{script} serta proses yang berbeda. Seperti yang sudah dijelaskan sebelumnya, komponen ini akan menembak permintaan HTTP pada \href{https://www.elastic.co/guide/en/elasticsearch/reference/current/cluster-nodes-stats.html}{\textit{Node Stats API}} yang telah disediakan \textit{Elastic Search} lalu melakukan transformasi bentuk data menjadi bentuk yang lebih sederhana dan sesuai kebutuhan. Komponen ini akan berjalan pada \textit{script} yang berbeda dikarenakan bahasa Python memiliki kekurangan dalam penanganan \textit{multithreading}. Komponen ini akan mengirimkan data yang sudah diolah ke komponen \textit{Predictor} melalui \textit{stream file}. Pendekatan ini dipilih karena sederhana dan mudah diimplementasikan. Khusus komponen ini, struktur kodenya tidak memakai sistem kelas dan hanya terdapat sebuah fungsi dan beberapa baris perintah untuk melakukan pemanggilan API, transformasi data dan pengiriman data ke \textit{stream file}.

\subsubsection{Komponen \textit{Predictor}}
Komponen \textit{Predictor} terdiri dari 3 buah kelas, yaitu sebagai berikut.
\begin{enumerate}
    \item \textbf{\textit{Predict Component}}
    
    Kelas ini berfungsi untuk menyimpan sebuah model ARIMA untuk sebuah variabel. Kelas ini memanfaatkan kakas pandas, statsmodels dan pmdarima untuk melakukan tanggung jawabnya.

    \item \textbf{\textit{Predict Component Factory}}
    
    Kelas ini berfungsi untuk membuat objek \textbf{\textit{Predict Component}} sebanyak variabel yang ada. 

    \item \textbf{\textit{Predict Component Storage}}
    
    Kelas ini berfungsi sebagai aggregator objek \textbf{\textit{Predict Component}} yang telah dibuat oleh \textbf{\textit{Predict Component Factory}}. Kelas ini juga berfungsi untuk meneruskan sebuah aksi kepada semua objek \textbf{\textit{Predict Component}} yang ada. Contohnya, dengan memanggil \textit{forecast} atau \textit{update data}, maka operasi akan diteruskan ke semua objek \textbf{\textit{Predict Component}}.

\end{enumerate}

\begin{figure}[h]
    \centering
    \includegraphics[width=0.8\textwidth]{chapter-4/predictor.png}
    \caption{Spesifikasi Kelas Penyusun Komponen \textit{Predictor}}
    \label{fig:predictor-spek}
\end{figure}

Secara umum, spesifikasi kelas bisa dilihat pada gambar \ref{fig:predictor-spek}. Kelas \textbf{\textit{Predict Component Storage}} akan membutuhkan \textbf{\textit{Predict Component Factory}} untuk membangun semua \textbf{\textit{Predict Component}} untuk setiap variabel yang ada. Setelah itu, terdapat operasi seperti meneruskan penambahan data serta meminta data prediksi ke setiap \textbf{\textit{Predict Component}}. Kelas ini akan digunakan oleh komponen \textit{Adaptive Control} untuk lebih lanjutnya.

\subsubsection{Komponen \textit{Rule Manager}}
Komponen \textit{Rule Manager} berfungsi untuk melakukan parsing terhadap file \textit{rule} yang telah diisi oleh pengguna serta menjadi aggregator untuk melakukan pengecekan \textit{rule} yang berlangsung serta memberi informasi data prediksi kapan saja yang dibutuhkan untuk melakukan pengecekan. Parsing komponen ini menggunakan format csv dan kondisi diekspresikan dengan sintaks python. Komponen ini akan menghasilkan sebuah objek \textit{Rule} yang akan digunakan oleh komponen \textit{Adaptive Control}. Agar terbayang, contoh dari \textit{file rule} dapat dilihat pada lampiran XXX. Spesifikasi dari kedua kelas tersebut dapat dilihat pada gambar \ref{fig:rule-spek}.

% TODO CONTOH RULE, masukin ke lampiran, trus tag kesini.

Sebuah \textit{rule} memiliki fungsi sebagai berikut.
\begin{enumerate}
    \item Memiliki sebuah kondisi yang akan dievaluasi dengan data prediksi pada waktu prediksi yang diinginkan. Contoh: kondisi \textit{throughput} untuk operasi X untuk 1 menit kedepan dan 5 menit kedepan lebih dari 1s, maka tingkatkan prosesor sebanyak 500m.
    \item Memiliki jumlah serta target kategori untuk diubah, dalam kasus ini pilihannya memori atau prosesor.
    \item Satuan untuk perubahan memori adalah dalam \textit{Mebibyte} atau MiB. Sedangkan untuk prosesor dalam satuan mili atau m.
    \item Sebuah \textit{rule} memiliki periode pengecekan sehingga tidak akan dicek secara terus menerus yang menyebabkan perubahan alokasi sumber daya terlalu cepat. Periode pengecekan dibuat dalam satuan sekon.
\end{enumerate}

\begin{figure}[h]
    \centering
    \includegraphics[width=0.8\textwidth]{chapter-4/rule.png}
    \caption{Spesifikasi Kelas Penyusun Komponen \textit{Rule Manager}}
    \label{fig:rule-spek}
\end{figure}

\subsubsection{Komponen \textit{Resource Controller}}

Komponen ini terdiri dari sebuah kelas. Seperti namanya, kelas ini berfungsi untuk menggunakan \textit{Kubernetes Client API} untuk mengubah alokasi sumber daya. Kelas ini diimplementasikan dengan sistem antrian, sehingga jika sejumlah rule aktif secara bersamaan, maka akan dijalankan secara berurutan. Terdapat sebuah fungsi \textit{tick} yang akan berfungsi untuk mengeksekusi antrian. Spesifikasi kelas ini dapat dilihat pada gambar \ref{fig:rc-spek}.

\begin{figure}[h]
    \centering
    \includegraphics[width=0.8\textwidth]{chapter-4/rc.png}
    \caption{Spesifikasi Kelas Penyusun Komponen \textit{Resource Controller}}
    \label{fig:rc-spek}
\end{figure}

\subsubsection{Komponen \textit{Adaptive Control}}

\subsubsection{Tampilan Implementasi}

\subsection{Eksperimen \textit{In-place Resource Resize} untuk \textit{pods Kubernetes}}

Pada bagian sebelumnya, diperlukan \textit{resizing} sumber daya yang dialokasikan untuk \textit{pods} yang sedang berjalan tanpa melakukan \textit{restart}. Untuk hal tersebut, perlu dilakukan eksperimen, berikut adalah rincian dari eksperimen yang telah dilakukan.

\subsubsection{Pendahuluan}

\textit{In-place Resource Resize} adalah fitur untuk mengubah ukuran sumber daya CPU dan memori yang dialokasikan untuk kontainer pada pod yang sedang berjalan tanpa harus me-\textit{restart} pod atau kontainernya. Sebuah \textit{node} Kubernetes mengalokasikan sumber daya untuk sebuah pod berdasarkan permintaannya, dan membatasi penggunaan sumber daya pod berdasarkan batasan yang ditentukan dalam kontainer-kontainer pod tersebut. Fitur ini baru hadir pada versi Kubernetes 1.27.0, dan, pada saat tugas akhir ini dikerjakan, masih dalam tahap \textit{alpha testing} dan pengembangan. Berdasarkan \parencite{kubeinplaceupdate2}, dokumentasi resmi dipublikasikan pada 30 Maret 2023 7:59 PM PST. Sedangkan berdasarkan \parencite{kubeinplaceupdate}, publikasi \textit{alpha testing} semenjak 13 Mei 2023.

\subsubsection{Pengerjaan Eksperimen}
Dalam melakukan eksperimen ini, dilakukan beberapa tahap sebagai berikut.

\begin{enumerate}
    \item Memastikan versi Kubernetes yang digunakan adalah versi 1.27.0 atau lebih baru pada \textit{client} dan \textit{server}.
    
    Pada saat itu, Kubernetes lokal yang dipakai adalah \textit{Docker Desktop Kubernetes} yang membatasi versi Kubernetes pada versi 1.25.9. Sehingga, dilakukan pengubahan server dengan menggunakan Minikube. Sayangnya, versi maksimal yang bisa dipakai oleh Minikube adalah Kubernetes versi 1.27.0-rc0, versi 1.27 yang paling pertama atau \textit{release candidate 0}. Untuk mengecek bisa dilihat pada \url{https://github.com/kubernetes/minikube/releases/tag/v1.30.0}. Dicoba juga untuk dipaksa menggunakan versi 1.27.1 maupun 1.27.3, namun hal tersebut gagal untuk dilakukan. Sehingga, untuk eksperimen ini, digunakan Kubernetes versi 1.27.0-rc0 melalui Minikube.

    \item Membuat \textit{deployment} yang akan digunakan untuk eksperimen.
    
    Konfigurasi \textit{deployment} berubah, karena untuk menggunakan fitur ini, tidak bisa membuat \textit{pods} dengan menggunakan tipe \textit{deployment} melainkan harus langsung membuat dengan tipe \textit{pod}. Terdapat juga beberapa konfigurasi baru yang perlu diatur.

    \item Mengeksekusi perintah untuk melakukan \textit{in-place resource resize}.
    
    Hal ini bisa dilakukan dengan mengeksekusi perintah \textit{patch pod}. Perintah ini bisa dilihat pada dokumentasi: \url{https://kubernetes.io/docs/tasks/configure-pod-container/resize-container-resources/}.

    Saat hal ini dijalankan terdapat pesan eror yang mengatakan bahwa fitur \textit{patch} tersebut hanya bisa dilakukan selain resource. Padahal, inisiasi eksperimen sudah disesuaikan dengan \textit{requirement} yang disebutkan pada dokumentasi.

    \item Mengecek detail informasi \textit{pods}
    
    Seharusnya, ketika mengecek detail informasi \textit{pods} akan terlihat bahwa resource yang digunakan sudah berubah dan terdapat informasi-informasi baru yang hanya muncul pada versi terbaru Kubernetes. Namun, hal ini tidak terjadi. Resource yang digunakan masih sama dengan sebelumnya. Dan hasil detail informasi sebuah \textit{pods} tidak memperlihatkan detail informasi terbaru yang sesuai dengan contoh pada dokumentasi.
\end{enumerate}

\subsubsection{Hasil Eksperimen}

Berdasarkan hasil eksperimen tersebut, dapat disimpulkan bahwa.

\begin{enumerate}
    \item Fitur \textit{in-place resource resize} belum bisa digunakan pada Kubernetes versi 1.27.0-rc0.
    \item Melakukan \textit{resource resize} tanpa melakukan \textit{restart} mungkin bisa dilakukan pada masa yang akan datang.
    \item Eksperimen ini bisa dicoba lagi pada masa yang akan datang. Mengingat tools kubernetes lokal masih belum bisa menggunakan versi \textit{alpha} terbaru. Dan, pada saat ini, fitur ini masih dalam tahap \textit{alpha testing} dan pengembangan.
\end{enumerate}

Sehingga, untuk saat ini, eksperimen ini tidak bisa dilakukan lebih jauh lagi. Dan oleh karena itu, sistem kontrol adaptif yang dibuat masih memakai sistem \textit{Rolling Update} untuk mengubah sumber daya alokasi. Namun, kedepannya, jika ingin diteruskan, besar kemungkinan hal ini bisa dilakukan karena dari Kubernetes sendiri sedang mengembangkan fitur tersebut.

\section{Pengujian}

Bagian ini akan menjelaskan beberapa skenario yang dilakukan untuk menguji \textit{autoscaler} dengan kontrol fleksibel. Pengujian akan dilakukan per komponen lalu dilanjutkan dengan satu sistem penuh. Setiap skenario pengujian akan dijelaskan tujuannya, skenario yang dilakukan, dan hasil pengujian yang didapatkan.

% \subsection{Pengujian X}

% \subsubsection{Tujuan Pengujian}

% \subsubsection{Skenario Pengujian}

% \subsubsection{Hasil Pengujian dan Analisis}

\input{chapters/chapter-4/06-pengujian-mf.tex}
\subsection{Pengujian Komponen \textit{Rule Manager}}
\subsection{Pengujian Komponen \textit{Predictor}}
\subsection{Pengujian Komponen \textit{Resource Controller}}
\input{chapters/chapter-4/07-pengujian-ac.tex}
\chapter{Penutup}

Bab Kesimpulan dan Saran akan menjadi bagian akhir dan penutup dari penelitian tugas akhir ini. Bab ini akan membahas kesimpulan yang berisi ketercapaian tujuan penelitian tugas akhir dengan permasalahan yang diselesaikan dalam penelitian tugas akhir. Selain itu, bab ini akan membahas saran yang dapat dilakukan untuk pengembangan atau penelitian selanjutnya.

\section{Kesimpulan}
Penelitian tugas akhir ini mengimplementasikan metode baru untuk melakukan \textit{autoscaling} yang disebut sebagai sistem kontrol fleksibel. Setelah dilakukan analisis, implementasi, dan pengujian, dapat diambil kesimpulan sebagai berikut.
\begin{enumerate}
    \item Melalui implementasi sistem kontrol fleksibel, pengguna dapat mengatur \textit{autoscaler} untuk melakukan \textit{scaling} berdasarkan variabel yang ada pada \textit{Elastic Search}.
    \item Dengan metode \textit{autoscaler} berbasis model prediksi, sistem dapat lebih baik dalam melakukan \textit{scaling} dibandingkan dengan \textit{autoscaler} sederhana yang memakai \textit{treshold} karena tidak terpengaruh oleh fluktuasi data.
    \item \textit{Autoscaler} menjadi lebih banyak ruang untuk melakukan \textit{scaling} karena tidak terbatas oleh sebuah angka \textit{treshold} melainkan oleh kondisi-kondisi yang disesuaikan dengan kebutuhan pemakai.
    \item Proses \textit{trial and error} hilang karena adanya sebab-akibat yang jelas pada alokasi sumber daya terhadap variabel \textit{throughput} pada operasi-operasi \textit{Elastic Search}. Pengguna hanya perlu menentukan standar \textit{throughput} pada operasi tertentu dan membuat kondisi yang keterhubungan untuk melakukan \textit{scaling}.
    \item \textit{Rolling Update} dapat dilakukan saat Kubernetes sudah merilis versi stabil dari 1.27 untuk melakukan \textit{In-pod resource resizing}.
\end{enumerate}

\section{Saran}
Adapun banyak kekurangan dan kelemahan yang ditemukan dalam penelitian tugas akhir ini. Berikut adalah beberapa saran yang dapat dilakukan untuk pengembangan atau penelitian selanjutnya.
\begin{enumerate}
    \item Melakukan permodelan dengan Bi-LSTM atau RNN untuk mengurangi waktu \textit{training}. Model ARIMA sangat memakan waktu saat \textit{training}.
    \item Melakukan pengembangan di bahasa lain yang lebih cepat dalam melakukan pemrosesan dibanding Python. Tentunya hal ini akan berpengaruh terhadap pembuatan kakas model statistik maupun \textit{machine learning} dikarenakan Python merupakan bahasa yang paling banyak digunakan untuk \textit{data science} dan \textit{machine learning}.
    \item Memasukkan sistem \textit{autoscaler} ke dalam \textit{cluster} Kubernetes seperti \textit{sidecar pods} agar memudahkan melakukan \textit{scaling Elastic Search} itu sendiri.
    \item Riset replikasi multi-node \textit{Elastic Search}. 
    \item Melakukan percobaan di \textit{cluster} Kubernetes dan \textit{Elastic Search} yang lebih besar dan lebih kompleks.
\end{enumerate}
%---------------------------------------------------------------%

% Daftar pustaka
\printbibliography

% Setting judul lampiran
\titlespacing*{\chapter}{0pt}{0pt}{0pt}
\titlespacing*{\section}{0pt}{0pt}{*1}

% Setting judul anak lampiran
\titleformat*{\section}{\bfseries}

\appendix
\renewcommand{\appendixname}{Lampiran}

\chapter{Ilustrasi Implementasi Arsitektur VeeR EL2}
\label{appendix:veer-el2-full}

\begin{figure}[h]
	\centering
	\begin{sideways}
		\includegraphics[width=1.3\textwidth]{chapter-3/veer-el2-full.jpg}
	\end{sideways}
	\caption{Ilustrasi Implementasi Arsiktektur VeeR EL2 \parencite{chip2024cores}}
	\label{fig:veer-el2-full}
\end{figure}

\chapter{Diagram dan \textit{Pseudocode} Algoritma}

\begin{figure}[H]
	\centering
	\includegraphics[width=0.7\textwidth]{chapter-3/path-finding-hopcroft.jpeg}
	\caption{Algoritma \ac{DFS} untuk \textit{pathfinding} \parencite{hopcroft1973algorithm}}
	\label{fig:hopcroft-dfs}
\end{figure}

\begin{algorithm}
	\makeatletter
	\renewcommand{\ALG@name}{Algoritma}
	\makeatother
	\caption{Prim \textit{Generator} Labirin}\label{alg:prim}
	\renewcommand{\algorithmicrequire}{\textbf{Masukan:}}
	\renewcommand{\algorithmicensure}{\textbf{Keluaran:}}
	\begin{algorithmic}[1]
		\Require $labirin$ dan $dimensi$
		\Ensure $labirin$ yang telah terbangun
		\State Inisialisasi $labirin$ penuh dengan halangan
		\State Buat halangan pertama jadi sel kosong
		\State Tambah semua halangan, disamping sel kosong, ke $list\_halangan$
		\While{$list\_halangan$ belum kosong}
		\State $halangan\_kini$ $\gets$ halangan acak di $list\_halangan$
		\If{$halangan\_kini$ hanya punya satu sel kosong didekatnya}
		\State Ubah $halangan\_kini$ menjadi sel kosong
		\State Masukkan halangan di sekitar $halangan\_kini$ ke $list\_halangan$
		\EndIf
		\State hapus $halangan\_kini$ dari $list\_halangan$
		\EndWhile
	\end{algorithmic}
\end{algorithm}


\begin{algorithm}
	\makeatletter
	\renewcommand{\ALG@name}{Algoritma}
	\makeatother
	\caption{\ac{DFS} pada labirin}\label{alg:dfs-sw}
	\renewcommand{\algorithmicrequire}{\textbf{Masukan:}}
	\renewcommand{\algorithmicensure}{\textbf{Keluaran:}}
	\begin{algorithmic}[1]
		\Require $baris$, $kolom$, dan $memori$
		\Ensure $langkah\_diambil$
		\State $memori \gets$ [$baris$,$kolom$]
		\If{[$baris$,$kolom$] = tujuan}
		\State return 0 \EndIf
		\State $langkah\_diambil$ $\gets$ NULL
		\For{$aksi$ dari [$gerak\_atas$, $gerak\_bawah$, $gerak\_kiri$, $gerak\_kanan$]}
		\State $kolom\_baru$, $baris\_baru$ $\gets$ lingkungan($aksi$, $kolom$, $baris$)
		\If{[$kolom\_baru$,$baris\_baru$] \textbf{tidak di} $memori$}
		\State $langkah\_DFS$ $\gets$ DFS($kolom\_baru$, $baris\_baru$, $memori$) + 1
		\If{$langkah\_diambil$ = NULL \textbf{or} $langkah\_diambil < langkah\_DFS$}
		\State $langkah\_diambil \gets langkah\_DFS$
		\EndIf
		\EndIf
		\EndFor
		\State return $langkah\_diambil$
	\end{algorithmic}
\end{algorithm}

\begin{algorithm}
	\makeatletter
	\renewcommand{\ALG@name}{Algoritma}
	\makeatother
	\caption{\ac{RL} menggunakan \textit{Q-Learning} diadaptasi dari \parencite{sutisna2023faraneq}}\label{alg:rl-qlearning}
	\renewcommand{\algorithmicrequire}{\textbf{Masukan:}}
	\renewcommand{\algorithmicensure}{\textbf{Keluaran:}}
	\begin{algorithmic}[1]
		\Require jumlah episode, \textit{learning rate ($\alpha$)}, dan \textit{discount factor ($\gamma$)}
		\Ensure \textit{Q-Table}
		\State Inisialisasi \textit{Q-Table} untuk setiap \textit{state} dan aksi
		\While{$banyak\_episode < jumlah\_episode$}
		\State $s_{t} \gets s_0$
		\While{$s_t \neq s_{terminal}$}
		\State Pilih aksi ($A$) untuk \textit{state} kini ($s_t$)
		\State Lakukan $A$, dapatkan \textit{reward} dan $s_{t+1}$ \label{algline:q-reward}
		\State Hitung nilai Q-Table indeks terkini \Comment{dengan Persamaan \ref{eq:q-learning}}
		\State $s_t \gets s_{t+1}$
		\EndWhile
		\State $banyak\_episode \gets banyak\_episode + 1$
		\EndWhile
	\end{algorithmic}
\end{algorithm}

\begin{algorithm}
	\makeatletter
	\renewcommand{\ALG@name}{Algoritma}
	\makeatother
	\caption{Desain fungsi \textit{reward}}\label{alg:rl-reward-function}
	\renewcommand{\algorithmicrequire}{\textbf{Masukan:}}
	\renewcommand{\algorithmicensure}{\textbf{Keluaran:}}
	\begin{algorithmic}[1]
		\Require $baris$, $kolom$, $memori$, dan aksi
		\Ensure $reward$
		\State $kolom\_baru$, $baris\_baru$ $\gets$ lingkungan($aksi$, $kolom$, $baris$)
		\If{[$kolom\_baru$,$baris$] = halangan} \label{algline:reward-c1}
		\State return $C_1$
		\EndIf
		\If{[$kolom\_baru$,$baris$] \textbf{ada di} $memori$} \label{algline:reward-c2}
		\State return $C_2$
		\EndIf
		\If{[$kolom\_baru$,$baris$] = tujuan} \label{algline:reward-c3}
		\State return $C_3$
		\EndIf
		\State return $C_4$ \label{algline:reward-c4}
	\end{algorithmic}
\end{algorithm}

\begin{algorithm}
	\makeatletter
	\renewcommand{\ALG@name}{Algoritma}
	\makeatother
	\caption{Ekstensi algoritma \ac{RL} dengan memoisasi pintar adopsi dari \parencite{mazaya2024reinforcement}}\label{alg:rl-qmemo}
	\renewcommand{\algorithmicrequire}{\textbf{Masukan:}}
	\renewcommand{\algorithmicensure}{\textbf{Keluaran:}}
	\begin{algorithmic}[1]
		\Require jumlah episode, \textit{learning rate} ($\alpha$), dan \textit{discount factor} ($\gamma$)
		\Ensure \textit{Q-Table}
		\State Inisialisasi \textit{Q-Table} untuk setiap \textit{state} dan aksi
		\While{$banyak\_episode < jumlah\_episode$}
		\State $s_{t} \gets s_0$
		\While{$s_t \neq s_{terminal}$}
		\State Pilih aksi ($A$) untuk \textit{state} kini ($s_t$)
		\State Lakukan $A$, dapatkan \textit{reward} dan $s_{t+1}$
		\State Hitung nilai Q-Table indeks terkini \Comment{dengan Persamaan \ref{eq:q-learning}}
		\State $s_t \gets s_{t+1}$
		\EndWhile
		\State $banyak\_episode \gets banyak\_episode + 1$
		\State $cr_t$ $\gets$ \textit{cumulative reward} dari \textit{Q-Table}
		\If{$cr_t\ >\ cr_{t-1}$}
		\State Ubah nilai $maxQ(s_{t+1},a))$ dari \textit{Q-Table} \label{algline:q-memo-equation} \Comment{dengan Persamaan \ref{eq:q-memo-overwrite}}
		\EndIf
		\State $cr_{t-1} \gets cr_t$
		\State \textit{memori Q-Table} $\gets$ \textit{Q-Table}
		\State $episode \gets episode + 1$
		\EndWhile
	\end{algorithmic}
\end{algorithm}

\begin{algorithm}
	\makeatletter
	\renewcommand{\ALG@name}{Algoritma}
	\makeatother
	\caption{Pembagian HW/SW \textit{co-design}}\label{alg:hw-sw-sep}
	\renewcommand{\algorithmicrequire}{\textbf{Masukan:}}
	\renewcommand{\algorithmicensure}{\textbf{Keluaran:}}
	\begin{algorithmic}[1]
		\Require jumlah episode, \textit{learning rate} ($\alpha$), dan \textit{discount factor} ($\gamma$)
		\Ensure \textit{Q-Table}
		\State \alghighlight{yellow!50}{Inisialisasi \textit{Q-Table} untuk setiap \textit{state} dan aksi \label{algline:initialize-q-table-hw} \Comment{\textbf{Perangkat keras}}}
		\State \alghighlight{yellow!50}{Simpan $\gamma$ dan $\alpha$ pada akselerator \Comment{\textbf{Perangkat keras}}}
		\While{$banyak\_episode < jumlah\_episode$}
		\State $s_{t} \gets s_0$
		\While{$s_t \neq s_{terminal}$}
		\State \alghighlight{yellow!50}{Pilih aksi ($A$) untuk \textit{state} kini ($s_t$) \label{algline:choose-action-hw}  \Comment{\textbf{Perangkat keras}}}
		\State Lakukan $A$, dapatkan \textit{reward} dan $s_{t+1}$
		\State \alghighlight{yellow!50}{Simpan $s_{t+1}$ pada akselerator \Comment{\textbf{Perangkat keras}}}
		\State \alghighlight{yellow!50}{Hitung nilai \textit{Q-Table} indeks terkini \label{algline:q-table-update-hw} \Comment{\textbf{Perangkat keras}}}
		\State $s_t \gets s_{t+1}$
		\EndWhile
		\State $banyak\_episode \gets banyak\_episode + 1$
		\State $cr_t$ $\gets$ \textit{cumulative reward} dari \textit{Q-Table}
		\If{$cr_t\ >\ cr_{t-1}$}
		\State Ubah nilai $maxQ(s_{t+1},a))$ dari \textit{Q-Table}
		\EndIf
		\State $cr_{t-1} \gets cr_t$
		\State \textit{memori Q-Table} $\gets$ \textit{Q-Table}
		\State $episode \gets episode + 1$
		\EndWhile
	\end{algorithmic}
\end{algorithm}


\chapter{Konfigurasi dan Hasil Timing Diagram dari Verilator}
\label{appendix:verilator}

Terdapat dua unit pengujian yang dilakukan untuk mengetahui akurasi yang dimiliki oleh akselerator: q.max dan q.update. Instruksi lain, digunakan dan diuji sekaligus untuk menguji kedua instruksi tersebut.

Pertama, pengujian q.max, dilakukan dengan membuat sebuah program pada bahasa C pada kode \ref{fig:verilator-qmax}.

\begin{figure}[H]
	\centering
	\begin{lstlisting}[language=C,escapechar=|,numbers=left,caption={Program bahasa C untuk pengujian q.max},label={fig:verilator-qmax},captionpos=b]
int main() {
  setAMax(4);
  storeQValue(10.2, 0);
  storeQValue(6.2, 1);
  storeQValue(3.2, 2);
  storeQValue(12.0, 3);
  getMax(0);
}
\end{lstlisting}
\end{figure}

Hasil dari program diatas, seharusnya menghasilkan indeks 3 sebagai jawaban dari fungsi getMax yang merupakan \ac{BSP} dari q.max. Berikut merupakan hasil sintesis \textit{timing diagram} untuk program \ref{fig:verilator-qmax}.

\begin{figure}[h]
	\centering
	\includegraphics[width=1\textwidth]{appendix/gtkwave-qmax.jpg}
	\caption{Hasil \textit{timing diagram} program q.max}
	\label{fig:gtkwave-qmax}
\end{figure}

Dapat diperhatikan pada gambar \ref{fig:gtkwave-qmax}, bahwa nilai keluaran \textit{max\_q\_index} yang merupakan representasi $a_{max}$ itu bernilai 3. Maka, hasil dari akselerator sudah sesuai untuk instruksi q.max.

Selanjutnya, untuk instruksi q.update, berikut merupakan program yang digunakan untuk menguji akurasi akselerator.


\begin{figure}[H]
	\centering
	\begin{lstlisting}[language=C,escapechar=|,numbers=left,caption={Program bahasa C untuk pengujian q.update},label={fig:verilator-qupdate},captionpos=b]
int main() {
  setAMax(4);
  setConstant(CONSTANT_TYPE_LEARNING_RATE, 0.4);
  setConstant(CONSTANT_TYPE_DISCOUNT_FACTOR, 0.28);
  storeQValue(10.2, 0);
  storeQValue(6.2, 1);
  storeQValue(3.2, 6);
  storeQValue(12.0, 12);
  setNextState(1);
  qUpdate(0, 2.5);
}
\end{lstlisting}
\end{figure}

Bila menggunakan persamaan \ref{eq:q-learning}, maka didapat nilai hasil dari kode \ref{fig:verilator-qupdate} adalah 8.26. Nilai tersebut, bila diubah ke heksadesimal maka akan bernilai 0x410432cb. Program \ref{fig:verilator-qupdate} kemudian dicoba dan didapatkan hasil pada gambar \ref{fig:gtkwave-qupdate}.

\begin{figure}[h]
	\centering
	\includegraphics[width=1\textwidth]{appendix/gtkwave-qupdate.jpg}
	\caption{Hasil \textit{timing diagram} program q.update}
	\label{fig:gtkwave-qupdate}
\end{figure}

Dapat dilihat, pada gambar \ref{fig:gtkwave-qupdate}, didapat hasil pada \textit{register} \textit{output update} yang bernilai 0x410432cb. Sehingga, akselerator sudah terimplementasikan secara akurat.

\chapter{Modul Akselerator}
\label{appendix:modul-akselerator}

\begin{figure}[H]
	\centering
	\begin{sideways}
		\includegraphics[width=1.3\textwidth]{chapter-3/max-seeker-unit.png}
	\end{sideways}
	\caption{Ilustrasi penggunaan \textit{Max Seeker Unit}}
	\label{fig:max-seeker-unit}
\end{figure}

\begin{figure}[H]
	\centering
	\begin{sideways}
		\includegraphics[width=1.5\textwidth]{chapter-3/q-table-updater.png}
	\end{sideways}
	\caption{Ilustrasi penggunaan \textit{Q-Table Updater}}
	\label{fig:q-table-updater}
\end{figure}

\chapter{Analisis Kestabilan Kontrol \textit{Cart Pole Balancing}}
\label{appendix:real-time}

Sistem \textit{cart pole} terdiri dari sebuah gerobak dengan massa \(M\) yang dapat bergerak secara horizontal tanpa gesekan, dan sebuah batang dengan panjang \(l\) dan massa \(m\) yang terpasang pada gerobak dan dapat berayun bebas. Berikut merupakan persamaan gerak dari masing-masing gerobak dan batang.

\begin{itemize}
	\item Persamaan Gerak Gerobak
	      \begin{flalign}
		      \label{eq:gerak-gerobak}
		      M\ddot{x} + m\ddot{x} + ml\ddot{\theta}\cos\theta - ml\dot{\theta}^2\sin\theta = F &  &
	      \end{flalign}
	\item Persamaan Gerak Batang
	      \begin{flalign}
		      \label{eq:gerak-batang}
		      ml^2\ddot{\theta} + ml\ddot{x}\cos\theta = mgl\sin\theta &  &
	      \end{flalign}
\end{itemize}

Untuk menurunkan matriks \(A\) dan \(B\), dilakukan linierisasi persamaan gerak di sekitar titik ekuilibrium \((x = 0, \theta = 0, \dot{x} = 0, \dot{\theta} = 0)\). Pada titik ini, \(\cos\theta \approx 1\) dan \(\sin\theta \approx \theta\). Dengan melakukan linierisasi dan menyederhanakan persamaan \ref{eq:gerak-gerobak} dan \ref{eq:gerak-batang}, didapat:

\begin{itemize}
	\item Persamaan Gerak Gerobak Linier:
	      \begin{flalign}
		      \label{eq:linier-gerak-gerobak}
		      (M + m)\ddot{x} + ml\ddot{\theta} = F &  &
	      \end{flalign}
	\item Persamaan Gerak Batang Linier:
	      \begin{flalign}
		      \label{eq:linier-batang-gerobak}
		      ml\ddot{x} + ml^2\ddot{\theta} = mgl\theta &  &
	      \end{flalign}
\end{itemize}

Dari persamaan linier ini, ruang keadaan dapat direpresentasikan dengan persamaan linier matriks \(\dot{\mathbf{x}} = A\mathbf{x} + B\mathbf{u}\). Dari persamaan \ref{eq:linier-gerak-gerobak} dan \ref{eq:linier-batang-gerobak} untuk \(\ddot{x}\) dan \(\ddot{\theta}\).

\begin{flalign}
	\label{eq:x-double-dot}
	\ddot{x} = \frac{F - ml\ddot{\theta}}{M + m} &  &
\end{flalign}

Substitusi \(\ddot{x}\) dari \ref{eq:x-double-dot} ke dalam persamaan \ref{eq:linier-batang-gerobak}.
\begin{flalign*}
	ml\left(\frac{F - ml\ddot{\theta}}{M + m}\right) + ml^2\ddot{\theta} = mgl\theta &  &
\end{flalign*}
\begin{flalign*}
	\Leftrightarrow \frac{mlF}{M + m} + ml^2\ddot{\theta} - \frac{m^2l^2\ddot{\theta}}{M + m} = mgl\theta &  &
\end{flalign*}
\begin{flalign*}
	\Leftrightarrow \ddot{\theta}\left(ml^2 - \frac{m^2l^2}{M + m}\right) = mgl\theta - \frac{mlF}{M + m} &  &
\end{flalign*}
\begin{flalign*}
	\Leftrightarrow \ddot{\theta}\left(\frac{ml^2(M + m) - m^2l^2}{M + m}\right) = mgl\theta - \frac{mlF}{M + m} &  &
\end{flalign*}
\begin{flalign*}
	\Leftrightarrow \ddot{\theta}\left(\frac{ml^2M}{M + m}\right) = mgl\theta - \frac{mlF}{M + m} &  &
\end{flalign*}
\begin{flalign*}
	\Leftrightarrow \ddot{\theta} = \frac{(M + m)mgl\theta - mlF}{ml^2M} &  &
\end{flalign*}
\begin{flalign*}
	\Leftrightarrow \ddot{\theta} = \frac{Mmg\theta + mmg\theta - mlF}{ml^2M} &  &
\end{flalign*}
\begin{flalign*}
	\Leftrightarrow \ddot{\theta} = \frac{m(M + m)g\theta - mlF}{ml^2M} &  &
\end{flalign*}
\begin{flalign}
	\label{eq:theta-double-dot}
	\Leftrightarrow \ddot{\theta} = \frac{(M + m)g\theta - \frac{F}{l}}{l(M + m)} &  &
\end{flalign}

Substitusi \ref{eq:theta-double-dot} ke persamaan \ref{eq:x-double-dot}:
\begin{flalign*}
	\ddot{x} = \frac{F - ml\left(\frac{(M + m)g\theta - \frac{F}{l}}{l(M + m)}\right)}{M + m} &  &
\end{flalign*}
\begin{flalign*}
	\Leftrightarrow \ddot{x} = \frac{F - m(M + m)g\theta + m\frac{F}{l}}{(M + m)} &  &
\end{flalign*}
\begin{flalign}
	\Leftrightarrow \ddot{x} = \frac{F - mg\theta(M + m)}{M + m} &  &
\end{flalign}

Dengan menggunakan notasi ruang keadaan \(\mathbf{x} = [x, \dot{x}, \theta, \dot{\theta}]^T\) dan \(\mathbf{u} = F\), matriks \(A\) dan \(B\) dapat disusun sebagai berikut:

\begin{flalign*}
	\dot{\mathbf{x}} = \begin{bmatrix}
		                   \dot{x}      \\
		                   \ddot{x}     \\
		                   \dot{\theta} \\
		                   \ddot{\theta}
	                   \end{bmatrix} = A \mathbf{x} + B u &  &
\end{flalign*}

\begin{flalign}
	\label{eq:a-raw}
	A = \begin{bmatrix}
		    0 & 1 & 0                         & 0 \\
		    0 & 0 & \frac{mg}{M + m}          & 0 \\
		    0 & 0 & 0                         & 1 \\
		    0 & 0 & \frac{(M + m)g}{l(M + m)} & 0
	    \end{bmatrix} &  &
\end{flalign}

\begin{flalign}
	\label{eq:b-raw}
	B = \begin{bmatrix}
		    0               \\
		    \frac{1}{M + m} \\
		    0               \\
		    \frac{1}{l(M + m)}
	    \end{bmatrix} &  &
\end{flalign}

Dengan substitusi parameter dari OpenAI Gym \parencite{towers2023gymnasium} \(M = 1,0 \, \text{kg}\), \(m = 0,1 \, \text{kg}\), \(l = 0,5 \, \text{m}\), dan \(g = 9,8 \, \text{m/s}^2\) ke \ref{eq:a-raw} dan \ref{eq:b-raw} maka didapat:

\begin{flalign}
	\label{eq:a-final}
	A = \begin{bmatrix}
		    0 & 1 & 0                                   & 0 \\
		    0 & 0 & \frac{0,1 \cdot 9,8}{1,1}           & 0 \\
		    0 & 0 & 0                                   & 1 \\
		    0 & 0 & \frac{1,1 \cdot 9,8}{0,5 \cdot 1,1} & 0
	    \end{bmatrix} = \begin{bmatrix}
		                    0 & 1 & 0     & 0 \\
		                    0 & 0 & 0,98  & 0 \\
		                    0 & 0 & 0     & 1 \\
		                    0 & 0 & 21,56 & 0
	                    \end{bmatrix} &  &
\end{flalign}

\begin{flalign}
	\label{eq:b-final}
	B = \begin{bmatrix}
		    0             \\
		    \frac{1}{1,1} \\
		    0             \\
		    \frac{1}{0,5 \cdot 1,1}
	    \end{bmatrix} = \begin{bmatrix}
		                    0     \\
		                    0,909 \\
		                    0     \\
		                    1,818
	                    \end{bmatrix} &  &
\end{flalign}


Untuk analisis kestabilan sistem kontrol diskrit, kita perlu melakukan konversi dari model kontinu ke model diskrit dengan menggunakan interval waktu sampling $T_s$. Lalu, apabila hasil pole dari berubah secara drastis dari nilai 1, maka sistem kontrol mulai tidak stabil. Berikut merupakan pole dari $T_s = 9 \text{ ms}$ dan $T_s = 0,2 \text{ ms}$.


\begin{itemize}
	\item $T_s = 9 \text{ ms}$
	      \label{eq-poles-ts-9}
	      \begin{flalign}
		      \text{Poles} =
		      \begin{bmatrix}
			      1,0427 \\
			      1,0000 \\
			      1,0000 \\
			      0,9591
		      \end{bmatrix} &  &
	      \end{flalign}
	\item $T_s = 0,2 \text{ ms}$
	      \label{eq-poles-ts-0.2}
	      \begin{flalign}
		      \text{Poles} =
		      \begin{bmatrix}
			      1,0009 \\
			      1,0000 \\
			      1,0000 \\
			      0,9991
		      \end{bmatrix} &  &
	      \end{flalign}
\end{itemize}

% Dari hasil \ref{eq-poles-ts-9} dan \ref{eq-poles-ts-0.2}, dapat dilihat bahwa untuk \( T_s = 9 \text{ ms} \), terdapat pole yang memiliki nilai lebih besar dari 1, yaitu 1,0427. Hal ini menunjukkan bahwa sistem kontrol mulai tidak stabil. Sedangkan untuk \( T_s = 0.2 \text{ ms} \), semua pole berada sangat dekat dengan nilai 1, menunjukkan bahwa sistem kontrol lebih stabil dibandingkan saat \( T_s = 9 \text{ ms} \).
Dari hasil \ref{eq-poles-ts-9} dan \ref{eq-poles-ts-0.2}, dapat dilihat bahwa untuk \( T_s = 9 \text{ ms} \), terdapat pole yang memiliki nilai lebih besar dari 1, yaitu 1,0427. Hal ini menunjukkan bahwa sistem kontrol mulai tidak stabil. Sedangkan untuk \( T_s = 0,2 \text{ ms} \), semua pole berada sangat dekat dengan nilai 1, menunjukkan sistem kontrol lebih stabil.



\end{document}
