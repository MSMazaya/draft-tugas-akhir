\clearpage
\chapter*{DAFTAR SINGKATAN DAN LAMBANG}
\addcontentsline{toc}{chapter}{DAFTAR SINGKATAN DAN LAMBANG}


\begin{table}[ht]
	\centering
	\begin{tabularx}{\textwidth}{>{\raggedright\arraybackslash}X >{\raggedright\arraybackslash}p{8cm} >{\centering\arraybackslash}X}
		SINGKATAN & \multicolumn{1}{c}{Nama}                  & \multicolumn{1}{>{\raggedright\arraybackslash}X}{Pemakaian pertama kali pada halaman} \\
		ALU       & \textit{Arithmetic Logic Unit}            & 26                                                                                    \\
		BSP       & \textit{Board Support Package}            & 25                                                                                    \\
		CISC      & \textit{Complex Instruction Set Computer} & 10                                                                                    \\
		DFS       & \textit{Depth First Search}               & 21                                                                                    \\
		EC        & \textit{Edge Computing}                   & 2                                                                                     \\
		FPGA      & \textit{Field Programmable Gate Array}    & 4                                                                                     \\
		HDL       & \textit{Hardware Design Language}         & 11                                                                                    \\
		IoT       & \textit{Internet of Things}               & 2                                                                                     \\
		ISA       & \textit{Instruction Set Architecture}     & 10                                                                                    \\
		LSU       & \textit{Load Store Unit}                  & 26                                                                                    \\
		LUTs      & \textit{Look Up Tables}                   & 12                                                                                    \\
		MSps      & \textit{Megasamples Per Second}           & 13                                                                                    \\
		MUX       & \textit{Multiplexer}                      & 32                                                                                    \\
		RAM       & \textit{Random Access Memory}             & 12                                                                                    \\
		RISC      & \textit{Reduced Instruction Set Computer} & 10                                                                                    \\
		RL        & \textit{Reinforcement Learning}           & 1                                                                                     \\
	\end{tabularx}
\end{table}

\begin{table}[ht]
	\centering
	\begin{tabularx}{\textwidth}{>{\raggedright\arraybackslash}X >{\raggedright\arraybackslash}p{8cm} >{\centering\arraybackslash}X}
		LAMBANG  & \multicolumn{1}{c}{Arti}                                      & \multicolumn{1}{>{\raggedright\arraybackslash}X}{Pemakaian pertama kali pada halaman} \\
		$A$      & Set aksi \textit{reinforcement learning}                      & 6                                                                                     \\
		$S$      & Set \textit{state} \textit{reinforcement learning}            & 6                                                                                     \\
		$\pi*$   & Strategi optimal \textit{reinforcement learning}              & 6                                                                                     \\
		$\gamma$ & Konstanta diskon pembelajaran \textit{reinforcement learning} & 6                                                                                     \\
		$\alpha$ & Konstanta pembelajaran \textit{reinforcement learning}        & 6                                                                                     \\
		$V$      & Fungsi nilai \textit{reinforcement learning}                  & 6                                                                                     \\
		$Q$      & Fungsi \textit{Q-Table}                                       & 7                                                                                     \\
		$\delta$ & Konstanta pembanding pada algoritma memoisasi                 & 23                                                                                    \\
	\end{tabularx}
\end{table}


\begin{acronym}
	\acro{RL}{\textit{Reinforcement Learning}}
	\acro{EC}{\textit{Edge Computing}}
	\acro{IoT}{\textit{Internet of Things}}
	\acro{FPGA}{\textit{Field Programmable Gate Array}}
	\acro{HDL}{\textit{Hardware Design Language}}
	\acro{ISA}{\textit{Instruction Set Architecture}}
	\acro{MSps}{\textit{Megasamples Per Second}}
	\acro{DFS}{\textit{Depth First Search}}
	\acro{BSP}{\textit{Board Support Package}}
	\acro{LSU}{\textit{Load Store Unit}}
	\acro{ALU}{\textit{Arithmetic Logic Unit}}
	\acro{RAM}{\textit{Random Access Memory}}
	\acro{MUX}{\textit{Multiplexer}}
	\acro{LUTs}{\textit{Look Up Tables}}
	\acro{CISC}{\textit{Complex Instruction Set Computer}}
	\acro{RISC}{\textit{Reduced Instruction Set Computer}}
\end{acronym}

\clearpage
