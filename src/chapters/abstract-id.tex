\clearpage
\chapter*{ABSTRAK}
\addcontentsline{toc}{chapter}{ABSTRAK}

\begin{center}
	\begin{singlespace}
		\large\bfseries\MakeUppercase{\thetitle}
		
		\normalfont\normalsize
		
		Oleh\\
		\bfseries{Muhammad Sulthan Mazaya \hspace{5mm} NIM: 13320028}
		
		\vspace{5mm}
		\large\bfseries{(Program Studi Teknik Fisika)}
		\vspace{5mm}
		
	\end{singlespace}
\end{center}

\begin{singlespace}
	\small
	\textit{Reinforcement Learning} (RL) merupakan salah satu kerangka pengembangan agen
	otonom yang marak digunakan. RL merupakan sebuah alternatif solusi pemodelan
	untuk sebuah permasalahan pada sebuah sistem yang terlalu kompleks untuk dibuat
	model secara matematis maupun algoritmik. Dengan demikian, RL marak
	digunakan pada permasalahan seperti robotika dan agen pengemudi otonom.
	Namun, meskipun sebuah alternatif yang tepat untuk banyak permasalahan, sering
	kali RL sulit untuk digunakan karena keterbatasan sumber daya perangkat keras.
	Hal ini disebabkan karena umumnya RL memerlukan sumber daya perangkat keras
	yang signifikan. Hal ini dapat menjadi hambatan saat ingin melakukan \textit{deployment
		model} RL pada komputer dengan sumber daya terbatas, seperti perangkat \textit{Internet
		of Things} (IoT) atau \textit{edge computing}. Oleh karena itu, pada penelitian ini dilakukan
	perancangan sebuah desain akselerator perangkat keras yang mampu membuat
	daya komputasi yang diperlukan untuk komputasi algoritma RL berkurang. Desain
	akselerator perangkat keras ini akan dilakukan pada sebuah \textit{Field Programmable
		Gate Array} dengan arsitektur RISC-V, sebuah arsitektur instruksi yang boleh
	digunakan secara terbuka, dalam bentuk \textit{co-processor}. Hasil yang telah dicapai
	pada tugas akhir ini berupa konfigurasi perangkat keras dan perangkat lunak,
	beserta desain dari perangkat lunak dan perangkat keras yang akan
	diimplementasikan.
	
	Kata kunci: \textit{reinforcement learning}, \textit{co-processor}, \textit{field programmable gate array}, RISC-V
\end{singlespace}
\clearpage
