\chapter{Studi Literatur}

Pada bab ini, akan diisi oleh studi literatur, hal-hal yang berkaitan dengan topik persoalan tugas akhir akan dipaparkan dalam bab ini guna untuk memberikan informasi mengenai dasar teori dan studi yang dipakai. Bab ini diharapkan membantu pembaca untuk mengerti dalam membaca penelitian tugas akhir ini.

\section{\textit{Information Retrieval}}

\section{Apache Lucene}

\section{Elastic Search}

\section{Java \textit{Virtual Machine}}

\section{\textit{Indexing}}

\section{\textit{Caching}}

\section{Kubernetes}

\subsection{Pods}

\subsection{\textit{Auto Scaler}}

\subsubsection{\textit{Horizontal Auto Scaler}}

\subsubsection{\textit{Vertical Auto Scaler}}

\subsection{\textit{Deployment File and Resource Limit}}

\subsection{Kubernetes Client}

\section{\textit{Metrics}}

\section{Pembelajaran Mesin}

\subsection{\textit{Supervised Learning}}

\subsection{\textit{Support Vector Machine}}