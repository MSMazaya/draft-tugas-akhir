\chapter{Penutup}

Bab Kesimpulan dan Saran akan menjadi bagian akhir dan penutup dari penelitian tugas akhir ini. Bab ini akan membahas kesimpulan yang berisi ketercapaian tujuan penelitian tugas akhir dengan permasalahan yang diselesaikan dalam penelitian tugas akhir. Selain itu, bab ini akan membahas saran yang dapat dilakukan untuk pengembangan atau penelitian selanjutnya.

\section{Kesimpulan}
Penelitian tugas akhir ini mengimplementasikan metode baru untuk melakukan \textit{autoscaling} yang disebut sebagai sistem kontrol fleksibel. Setelah dilakukan analisis, implementasi, dan pengujian, dapat diambil kesimpulan sebagai berikut.
\begin{enumerate}
    \item Melalui implementasi sistem kontrol fleksibel, pengguna dapat mengatur \textit{autoscaler} untuk melakukan \textit{scaling} berdasarkan variabel yang ada pada \textit{Elastic Search}.
    \item Dengan metode \textit{autoscaler} berbasis model prediksi, sistem dapat lebih baik dalam melakukan \textit{scaling} dibandingkan dengan \textit{autoscaler} sederhana yang memakai \textit{treshold} karena tidak terpengaruh oleh fluktuasi data.
    \item \textit{Autoscaler} menjadi lebih banyak ruang untuk melakukan \textit{scaling} karena tidak terbatas oleh sebuah angka \textit{treshold} melainkan oleh kondisi-kondisi yang disesuaikan dengan kebutuhan pemakai.
    \item Proses \textit{trial and error} hilang karena adanya sebab-akibat yang jelas pada alokasi sumber daya terhadap variabel \textit{throughput} pada operasi-operasi \textit{Elastic Search}. Pengguna hanya perlu menentukan standar \textit{throughput} pada operasi tertentu dan membuat kondisi yang keterhubungan untuk melakukan \textit{scaling}.
    \item \textit{Rolling Update} dapat dilakukan saat Kubernetes sudah merilis versi stabil dari 1.27 untuk melakukan \textit{In-pod resource resizing}.
\end{enumerate}

\section{Saran}
Adapun banyak kekurangan dan kelemahan yang ditemukan dalam penelitian tugas akhir ini. Berikut adalah beberapa saran yang dapat dilakukan untuk pengembangan atau penelitian selanjutnya.
\begin{enumerate}
    \item Melakukan permodelan dengan Bi-LSTM atau RNN untuk mengurangi waktu \textit{training}. Model ARIMA sangat memakan waktu saat \textit{training}.
    \item Melakukan pengembangan di bahasa lain yang lebih cepat dalam melakukan pemrosesan dibanding Python. Tentunya hal ini akan berpengaruh terhadap pembuatan kakas model statistik maupun \textit{machine learning} dikarenakan Python merupakan bahasa yang paling banyak digunakan untuk \textit{data science} dan \textit{machine learning}.
    \item Memasukkan sistem \textit{autoscaler} ke dalam \textit{cluster} Kubernetes seperti \textit{sidecar pods} agar memudahkan melakukan \textit{scaling Elastic Search} itu sendiri.
    \item Riset replikasi multi-node \textit{Elastic Search}. 
    \item Melakukan percobaan di \textit{cluster} Kubernetes dan \textit{Elastic Search} yang lebih besar dan lebih kompleks.
\end{enumerate}