\chapter{Penutup}

Bab Kesimpulan dan Saran akan menjadi bagian penutup dari penelitian tugas akhir ini. Dalam bab ini, akan disajikan kesimpulan yang mencerminkan pencapaian tujuan penelitian beserta masalah-masalah yang telah diatasi selama penelitian berlangsung. Selain itu, akan dibahas pula saran-saran yang dapat diimplementasikan untuk pengembangan lebih lanjut atau penelitian berikutnya.

\section{Kesimpulan}

Pada penelitian ini, telah dikembangkan sebuah akselerator perangkat keras yang dapat digunakan untuk melakukan komputasi algoritma \ac{RL} secara efisien. Berikut merupakan poin-poin utama dari kesimpulan riset ini.

\begin{enumerate}
	\item Memoisasi pada algoritma \textit{Q-Learning} dapat digunakan untuk menghasilkan sebuah pembelajaran \ac{RL} yang stabil. Teknik ini, dapat digunakan untuk mengatasi inkonsistensi dari pembelajaran untuk agen yang memerlukan aksi diskrit.
	\item Performa kecepatan akselerator berhasil dikembangkan dan memiliki performa kecepatan yang mengalahkan performa kecepatan perangkat lunak. Tren perkembangan pembelajaran menggunakan akselerator adalah $y = 5.494,05x + 7.364,74$ dengan $y$ adalah \textit{clock cycles} yang diperlukan dan $x$ adalah iterasi pembelajaran. Tren perkembangan \textit{inference} pada menggunakan akselerator adalah $y = 120,36x - 12,55$ dengan $y$ adalah \textit{clock cycles} yang diperlukan dan $x$ adalah banyak aksi dalam lingkungan \ac{RL}.
	\item Hasil utilisasi sumber daya akselerator dapat bersaing dengan \textit{state of the art} dari penelitian yang sudah dilakukan sebelumnya untuk resolusi 32-bit dengan 759 \ac{LUTs}, 1.259 \textit{flip-flops}, dan 1 blok \ac{RAM}.
	\item Akselerator yang didesain, dapat menjawab tantangan keperluan \textit{real time} dari hasil uji pada permasalahan \textit{cart pole balancing} dengan waktu eksekusi hanya sebesar 0,2 milisekon.
\end{enumerate}

\section{Saran}
Terdapat beberapa saran tentang hal-hal yang dapat ditingkatkan pada penelitian selanjutnya.
\begin{enumerate}
	\item Penggunaan unit \textit{out-of-pipe} berakibatkan instruksi tidak dapat dioptimasi secara \textit{pipelined} pada tingkat \textit{processor stages}. Sehingga, dapat dipertimbangkan untuk membuat \textit{pipe} terpisah pada \textit{execute} dan \textit{memory stage}.
	\item Implementasi blok \ac{RAM} internal seharusnya dapat dipindahkan kepada \ac{RAM} yang tertanam pada \textit{processor}. Sehingga, tidak terjadi redundansi pembuatan modul \ac{LSU}.
	\item \textit{Register-register} pada \textit{Q-Table} \ac{LSU}, dapat dibuat sebagai blok-blok kecil \ac{RAM} sehingga dapat dimuat secara lebih efisien. Salah satu contohnya adalah \textit{Max Seeker Unit} yang berpotensi untuk melakukan pembandingan nilai maksimal dalam satu \textit{clock cycle}.
	\item Unit \textit{floating addition unit} dan \textit{floating multiplyer unit} dapat dibuat \textit{superscalar} sehingga komputasi tidak perlu menunggu sampai 7 tahap.
\end{enumerate}
