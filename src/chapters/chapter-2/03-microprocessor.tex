\section{\textit{Microprocessor}}

\textit{Microprocessor} merupakan \textit{processor} yang dibangun di sebuah \textit{chip} \parencite{sarah2021digital}. Arsitektur dari sebuah \textit{microprocessor}, biasa disebut sebagai \textit{microarchitecture}, itu didefinisikan menggunakan sebuah \acf{ISA}. \ac{ISA} merupakan sebuah pedoman pembacaan instruksi yang digunakan oleh sebuah \textit{processor}. Jenis-jenis \ac{ISA} untuk \textit{processor} ini terbagi menjadi dua, yaitu \ac{CISC} dan \ac{RISC}.

\ac{CISC} adalah jenis arsitektur ISA yang menggunakan instruksi-instruksi kompleks dan beragam. Arsitektur ini bertujuan untuk mengurangi jumlah instruksi per program, memperkecil ukuran kode program, dan meminimalisasi akses ke memori dengan menyediakan instruksi-instruksi yang dapat melakukan banyak operasi dalam satu siklus. Contoh prosesor dengan arsitektur \ac{CISC} adalah prosesor Intel x86.

Sebaliknya, \ac{RISC} adalah jenis arsitektur ISA yang menggunakan seperangkat instruksi yang lebih sederhana dan lebih sedikit. Arsitektur ini dirancang untuk melakukan operasi dengan lebih cepat dan efisien dengan menjalankan setiap instruksi dalam satu siklus mesin. Contoh prosesor dengan arsitektur \ac{RISC} adalah ARM, PowerPC, dan RISC-V.

RISC-V adalah salah satu jenis arsitektur \ac{RISC} yang menonjol karena sifatnya yang terbuka dan bebas royalti. Dikembangkan oleh para peneliti di University of California, Berkeley, RISC-V dirancang dengan tujuan untuk menyediakan arsitektur yang fleksibel, hemat biaya, dan bebas dari batasan kepemilikan yang sering dihadapi oleh arsitektur \ac{ISA} lainnya \parencite{asanovic2014risc}.

Salah satu keunggulan utama dari RISC-V adalah keterbukaannya. Spesifikasi lengkap dari arsitektur ini tersedia secara publik, memungkinkan siapa saja untuk mengimplementasikan, memodifikasi, dan mendistribusikan perangkat keras dan perangkat lunak berbasis RISC-V tanpa harus membayar biaya lisensi. Sehingga, RISC-V ideal untuk digunakan untuk melakukan pengembangan \textit{processor} dengan ekstensi khusus.

Selanjutnya, pengembangan \textit{processor} khusus tersebut dapat dilakukan menggunakan \acl{FPGA} yang akan dijelaskan pada subbab selanjutnya.
