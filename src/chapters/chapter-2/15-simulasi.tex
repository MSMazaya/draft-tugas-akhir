\section{Simulasi dengan \textit{Elastic Search Benchmarking}}

\textit{Elastic Search Benchmarking} adalah metode yang digunakan untuk melakukan simulasi beban pada \textit{Elastic Search}. Dengan menggunakan \textit{Elastic Search Benchmarking}, pengguna dapat memprediksi jumlah beban yang dapat ditangani oleh \textit{Elastic Search} dan melakukan pencarian terhadap \textit{bottleneck} untuk diperbaiki.

\textit{Elastic Search Benchmarking} memiliki banyak cara pengujian seperti uji beban, uji rentang kisaran, uji keseimbangan, dan uji kestabilan. Uji beban bertujuan untuk mengukur jumlah beban yang dapat ditangani pada suatu waktu tertentu. Uji rentang kisaran dapat digunakan untuk membandingkan kinerja dari setiap rentang data pada variasi yang ada. Uji keseimbangan bertujuan untuk membandingkan kinerja sistem yang memiliki banyak \textit{node} pada sebuah \textit{cluster}. Terakhir, uji kestabilan digunakan untuk menguji kinerja sistem dalam penggunaan waktu yang lama.

\textit{Elastic Search Benchmarking} dapat digunakan untuk memprediksi besar biaya yang diperlukan untuk menjalankan \textit{Elastic Search} pada tingkat beban tertentu. Tak hanya itu, \textit{Elastic Search Benchmarking} dapat digunakan untuk mengidentifikasi titik lemah dan memberikan informasi yang berguna untuk meningkatkan kinerja sistem.

Salah satu tools untuk melakukan \textit{Elastic Search Benchmarking} adalah Rally. Tools ini dapat digunakan untuk mengukur kinerja \textit{Elastic Search} dalam berbagai situasi, termasuk pada lingkungan yang kompleks dan besar. Selain itu, Rally juga menyediakan sejumlah besar skenario pengujian kinerja bawaan yang dapat digunakan untuk menguji kinerja \textit{Elastic Search} dengan berbagai konfigurasi dan skenario pengujian yang berbeda. 