\section{\textit{Elastic Search}}
% \textit{Elastic Search} adalah aplikasi mesin pencarian RESTful.  \textit{Elastic Search} diciptakan sebagai pembungkus dan inovasi dari Apache Lucene yang sekedar hanya \textit{library} karena aplikasi yang beredar saat ini tidak hanya dibuat di atas Java dan membutuhkan fleksibilitas yang tinggi, sedangkan, Apache Lucene terkenal sangat sulit untuk orang awam yang tidak memahami istilah-istilah dan \textit{information retrieval}. \textit{Elastic Search} sendiri dibuat menggunakan Java namun menggunakan \textit{Application Programming Interface} RESTful melalui protokol HTTP sehingga aplikasi dengan bahasa apapun dapat dengan mudah menggunakan aplikasi ini. Tidak hanya itu, API dari \textit{Elastic Search} ini juga sudah sangat dipermudah sehingga pemakai tidak perlu mengetahui istilah-istilah dalam Apache Lucene. Sehingga, \textit{Elastic Search} ini sangat dekat dengan proses umum pada data seperti menyimpan, membaharui, menghapus, pengindeksan, pencarian, dan sebagainya.

\textit{Elastic Search} merupakan sebuah perangkat lunak \textit{open-source} yang ditulis menggunakan bahasa pemrograman Java. \textit{Elastic Search} dibangun di atas Apache Lucene. \textit{Elastic Search} menyimpan data secara terpusat untuk pencarian secepat kilat dan relevan \parencite{elasticsearchorigin}. Selain Apache Lucene, \textit{Elastic Search} memanfaatkan teknologi-teknologi lain untuk meningkatkan fungsionalitas dan performa, seperti Apache Hadoop untuk big data processing, Apache Spark untuk data analytics, dan Apache Storm untuk real-time stream processing.

\textit{Elastic Search} didesain sebagai sebuah sistem terdistribusi, yang berarti data yang disimpan pada \textit{Elastic Search} akan didistribusikan ke beberapa \textit{node} atau lebih dikenal sebagai \textit{sharding}, sehingga memungkinkan untuk meningkatkan performa, skalabilitas, dan ketahanan pada sistem. Sistem terdistribusi pada \textit{Elastic Search} dapat diatur dan dikonfigurasi agar dapat dijalankan pada beberapa \textit{node} yang terpisah atau pada \textit{cluster} yang terhubung, yang memungkinkan pengguna untuk menyimpan data yang sangat besar dan menjalankan query secara paralel pada beberapa node pada waktu yang bersamaan.

\textit{Elastic Search} dibuat untuk memudahkan pengguna mengakses data dan melakukan pencarian pada data yang besar dan kompleks dengan cepat dan efisien. Meskipun Apache Lucene telah menyediakan fitur-fitur yang bagus untuk \textit{indexing} dan \textit{searching}, tetapi Apache Lucene lebih fokus pada teknologi inti dan pengembangan secara \textit{information retrieval} seperti mengoptimisasi dan pengembangan \textit{indexing} dan \textit{searching}. Hal tersebut menyebabkan penggunaan Lucene memerlukan banyak pengaturan dan konfigurasi tambahan untuk bisa diintegrasikan dengan aplikasi yang lebih besar. Dalam hal ini, \textit{Elastic Search} hadir sebagai sebuah solusi yang lebih terintegrasi, mudah digunakan, dan dapat diatur secara fleksibel. \textit{Elastic Search} memanfaatkan Lucene sebagai mesin pencari, tetapi menambahkan banyak fitur-fitur dan fungsionalitas tambahan untuk meningkatkan performa dan kemudahan penggunaan. Selain itu, \textit{Elastic Search} dirancang sebagai aplikasi dengan sistem terdistribusi dan \textit{scalable} yang memungkinkan data terdistribusi di beberapa node atau \textit{sharding}, sehingga memungkinkan \textit{Elastic Search} untuk mengatasi masalah data yang sangat besar dan kompleks secara efektif dan efisien. Sedangkan, Lucene hanya pada batas kakas atau \textit{library}.

\textit{Elastic Search} dapat digunakan dengan protokol HTTP dan REST API. Dalam penggunaan dengan protokol HTTP, \textit{Elastic Search} menyediakan endpoint API RESTful HTTP yang dapat diakses oleh pengguna dengan memakai klien HTTP, seperti perintah cURL, \textit{Postman} atau \textit{web browser}. Pengguna dapat membuat permintaan HTTP seperti GET, POST, PUT, DELETE ke API endpoint melalui klien HTTP, dan \textit{Elastic Search} akan memberikan respon sesuai dengan permintaan.

Dalam \textit{Elastic Search} terdapat beberapa operasi yang dapat dilakukan, diantaranya:
\begin{enumerate}
    \item \textit{Index}
    
    Operasi ini akan dijelaskan secara khusus pada bagian selanjutnya, \ref{sec:index}.

    \item \textit{Get}
    
    Operasi \textit{Get} digunakan untuk mengambil dokumen individual berdasarkan ID-nya dari indeks tertentu.
    
    \item \textit{Query}
    
    \textit{Query} digunakan untuk melakukan pencarian dan pengambilan data yang sesuai dengan kriteria tertentu. \textit{Elasticsearch} menyediakan berbagai jenis query seperti pencarian teks lengkap, pencocokan kata kunci, pencocokan frasa, pencarian fuzzy, dan lain-lain.
    
    \item \textit{Fetch}
    
    \textit{Fetch} adalah proses pengambilan dokumen lengkap dari indeks setelah melakukan query. Saat ditemukan, \textit{Elasticsearch} mengambil dokumen dari indeks dan akan digunakan sebagai respon kepada pengguna.
    
    \item \textit{Scroll}
    
    \textit{Scroll} adalah mekanisme pengambilan dokumen yang banyak dari hasil pencarian tanpa perlu mengirimkan \textit{query} ulang. Hal ini menyebabkan pengambilan dokumen dapat menjadi lebih efisien dalam beberapa permintaan, namun, dalam beberapa kasus, bisa menjadi lebih lama untuk mengambil semua dokumen yang relevan.
    
    \item \textit{Suggest}
    
    \textit{Suggest} adalah mekanisme untuk memberikan saran atau \textit{autocompletion} saat pengguna memasukkan kata kunci atau frasa. Biasanya digunakan untuk \textit{autocompletion}, pengoreksi kesalahan pengejaan, dan saran pencarian lainnya.
    
    \item \textit{Bulk}
    
    \textit{Bulk} adalah operasi yang digunakan untuk memasukkan atau memperbarui beberapa dokumen dalam satu permintaan.
    
    \item \textit{Flush}
    
    \textit{Flush} adalah operasi yang digunakan untuk mengosongkan memori \textit{cache} dan menulis data yang tertunda ke disk. Operasi ini memastikan bahwa data yang ditulis telah disimpan secara permanen di indeks.
    
    \item \textit{Refresh}
    
    \textit{Refresh} adalah operasi yang digunakan untuk membuat perubahan yang terjadi pada indeks secara terlihat dan dapat dicari. Saat melakukan operasi indeks seperti menambahkan atau menghapus dokumen, perubahan tersebut tidak langsung terlihat oleh pencarian hingga dilakukan operasi refresh.
\end{enumerate}