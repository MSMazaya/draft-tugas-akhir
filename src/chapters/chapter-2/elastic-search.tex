\section{\textit{Elastic Search}}
% \textit{Elastic Search} adalah aplikasi mesin pencarian RESTful.  \textit{Elastic Search} diciptakan sebagai pembungkus dan inovasi dari Apache Lucene yang sekedar hanya \textit{library} karena aplikasi yang beredar saat ini tidak hanya dibuat di atas Java dan membutuhkan fleksibilitas yang tinggi, sedangkan, Apache Lucene terkenal sangat sulit untuk orang awam yang tidak memahami istilah-istilah dan \textit{information retrieval}. \textit{Elastic Search} sendiri dibuat menggunakan Java namun menggunakan \textit{Application Programming Interface} RESTful melalui protokol HTTP sehingga aplikasi dengan bahasa apapun dapat dengan mudah menggunakan aplikasi ini. Tidak hanya itu, API dari \textit{Elastic Search} ini juga sudah sangat dipermudah sehingga pemakai tidak perlu mengetahui istilah-istilah dalam Apache Lucene. Sehingga, \textit{Elastic Search} ini sangat dekat dengan proses umum pada data seperti menyimpan, membaharui, menghapus, pengindeksan, pencarian, dan sebagainya.

\textit{Elastic Search} merupakan sebuah perangkat lunak \textit{open-source} yang ditulis menggunakan bahasa pemrograman Java. \textit{Elastic Search} dibangun di atas Apache Lucene. \textit{Elastic Search} menyimpan data secara terpusat untuk pencarian secepat kilat dan relevan \parencite{elasticsearchorigin}. Selain Apache Lucene, \textit{Elastic Search} juga memanfaatkan teknologi-teknologi lain untuk meningkatkan fungsionalitas dan performa, seperti Apache Hadoop untuk big data processing, Apache Spark untuk data analytics, dan Apache Storm untuk real-time stream processing.

\textit{Elastic Search} didesain sebagai sebuah sistem terdistribusi, yang berarti data yang disimpan pada \textit{Elastic Search} akan didistribusikan ke beberapa \textit{node} atau lebih dikenal sebagai \textit{sharding}, sehingga memungkinkan untuk meningkatkan performa, skalabilitas, dan ketahanan pada sistem. Sistem terdistribusi pada \textit{Elastic Search} dapat diatur dan dikonfigurasi agar dapat dijalankan pada beberapa \textit{node} yang terpisah atau pada \textit{cluster} yang terhubung, yang memungkinkan pengguna untuk menyimpan data yang sangat besar dan menjalankan query secara paralel pada beberapa node pada waktu yang bersamaan.

\textit{Elastic Search} dibuat untuk memudahkan pengguna mengakses data dan melakukan pencarian pada data yang besar dan kompleks dengan cepat dan efisien. Meskipun Apache Lucene telah menyediakan fitur-fitur yang bagus untuk \textit{indexing} dan \textit{searching}, tetapi Apache Lucene lebih fokus pada teknologi inti dan pengembangan secara \textit{information retrieval} seperti mengoptimisasi dan pengembangan \textit{indexing} dan \textit{searching}. Hal tersebut menyebabkan penggunaan Lucene memerlukan banyak pengaturan dan konfigurasi tambahan untuk bisa diintegrasikan dengan aplikasi yang lebih besar. Dalam hal ini, \textit{Elastic Search} hadir sebagai sebuah solusi yang lebih terintegrasi, mudah digunakan, dan dapat diatur secara fleksibel. \textit{Elastic Search} memanfaatkan Lucene sebagai mesin pencari, tetapi menambahkan banyak fitur-fitur dan fungsionalitas tambahan untuk meningkatkan performa dan kemudahan penggunaan. Selain itu, \textit{Elastic Search} dirancang sebagai aplikasi dengan sistem terdistribusi dan \textit{scalable} yang memungkinkan data terdistribusi di beberapa node atau \textit{sharding}, sehingga memungkinkan \textit{Elastic Search} untuk mengatasi masalah data yang sangat besar dan kompleks secara efektif dan efisien. Sedangkan, Lucene hanya pada batas kakas atau \textit{library}.

\textit{Elastic Search} dapat digunakan dengan protokol HTTP dan REST API. Dalam penggunaan dengan protokol HTTP, \textit{Elastic Search} menyediakan endpoint API RESTful HTTP yang dapat diakses oleh pengguna dengan memakai klien HTTP, seperti perintah cURL, \textit{Postman} atau \textit{web browser}. Pengguna dapat membuat permintaan HTTP seperti GET, POST, PUT, DELETE ke API endpoint melalui klien HTTP, dan \textit{Elastic Search} akan memberikan respon sesuai dengan permintaan.