\section{Tujuan dan Sasaran}

Dengan menggunakan latar belakang serta permasalahan sebagai sumber acuan, maka ditentukan tujuan dari tugas akhir ini adalah agar terbentuknya sebuah desain akselerator perangkat keras yang dapat digunakan untuk melakukan komputasi algoritma RL secara lebih efisien dan efektif. Adapun sasaran dari penelitian ini adalah:

\begin{enumerate}
	\item Mengembangkan algoritma \ac{RL} untuk yang mampu melakukan pembelajaran secara konsisten untuk digunakan sebagai pengujian akselerator.
	\item Mendesain modul-modul perangkat keras pada \textit{processor} RISC-V untuk akselerator algoritma \ac{RL}.
	\item Melakukan sintesis hasil desain akselerator perangkat keras untuk mendapatkan hasil utilisasi sumber daya akselerator.
	\item Membuat uji simulasi sebagai metode untuk menguji hasil implementasi akselerator perangkat keras untuk kegunaan \textit{real time}.
\end{enumerate}
