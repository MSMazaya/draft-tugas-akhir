\section{Tujuan dan Sasaran}

Dengan menggunakan latar belakang serta permasalahan sebagai sumber acuan, maka ditentukan tujuan dari tugas akhir ini adalah agar terbentuknya sebuah desain akselerator perangkat keras yang dapat digunakan untuk melakukan komputasi algoritma RL secara lebih efisien dan efektif. Adapun sasaran dari penelitian ini adalah:

\begin{enumerate}
	\item Mendesain modul-modul perangkat keras maupun \textit{co-processor} RISC-V sebagai akselerator perangkat keras algoritma \ac{RL}.
	\item Mengimplementasikan modul-modul perangkat keras yang telah dirancang pada \ac{FPGA}.
	\item Membuat \textit{Board Support Package} untuk mengakses akselerator yang sudah dibangun.
	\item Membuat uji simulasi sebagai metode untuk menguji hasil implementasi akselerator perangkat keras menggunakan kasus \ac{RL} spesifik.
\end{enumerate}
