\section{Batasan Masalah dan Asumsi}
\label{sec:batasan-masalah}

Untuk memenuhi tujuan dan sasaran pada perancangan tugas akhir ini, beberapa batasan yang digunakan antara lain:

\begin{enumerate}
	\item Akselerator perangkat keras akan menggunakan arsitektur RISC-V.
	\item Modul perangkat keras akan ditulis dengan bahasa Verilog dan SystemVerilog pada \acf{FPGA}.
	\item Hasil desain akselerator perangkat keras tidak akan dicetak sebagai \textit{Application Specific Integrated Circuits}.
\end{enumerate}

Testing hasil akselerator perangkat keras akan dilakukan dengan kasus spesifik algoritma \ac{RL} untuk agen perencana jalur. Asumsi yang digunakan pada penelitian tugas akhir ini adalah:

\begin{enumerate}
	\item Simulator untuk pengujian sistem \textit{real time} menggunakan emulasi waktu \textit{delay} sudah merepresentasikan contoh implikasi waktu \textit{delay} pada skenario sesungguhnya.
	\item \textit{Clock cycle} pada \ac{FPGA} memiliki frekuensi yang konstan dan stabil.
\end{enumerate}
