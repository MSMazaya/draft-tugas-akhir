\section{Sistematika Pembahasan}

Konten dari Tugas Akhir ini akan dibagi menjadi lima bab sebagai berikut.
\begin{enumerate}
    \item Pendahuluan
    \item Studi Literatur
    \item Analisis Masalah dan Rancangan Solusi
    \item Implementasi dan Pengujian
    \item Kesimpulan dan Saran
\end{enumerate}

Pada Bab I akan dijelaskan gagasan utama dari tugas akhir ini yang berisi dari latar belakang, rumusan masalah, tujuan, batasan, metodologi hingga sistematika pembahasan.

Selanjutnya, Bab II akan menjelaskan hasil studi literatur yang berkaitan dengan pengerjaan tugas akhir ini. Bab II ini berisi tentang pemahaman dasar seputar topik yang akan dibahas pada tugas akhir ini.

Pada Bab III akan dijelaskan ulang masalah serta latar belakang untuk menyusun rancangan solusi. Di bab ini juga akan dipaparkan beberapa rancangan solusi yang kemudian akan dijelaskan lebih lanjut dan dipilih sebagai topik yang akan diimplementasikan pada bab selanjutnya.

% Bab IV ...

% Bab V akan menjadi penutup pada tugas akhir ini. Konten pada bab ini akan menjelaskan jawaban terhadap rumusan masalah pada bab I. Pada bab ini juga akan disebutkan saran-saran perbaikan yang bisa dipakai untuk penelitian berikutnya. Bab ini akan menyimpulkan hasil implementasi dan rancangan solusi terhadap masalah yang sudah diidentifikasi.