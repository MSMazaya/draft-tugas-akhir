\chapter{Pengujian Sistem}

Pada pengujian performa akselerator \ac{RL}, terdapat tiga komponen yang penting yang perlu ditinjau:

\begin{enumerate}
	\item Akurasi algoritma dan akselerator perangkat keras.
	\item Utilisasi sumber daya pada akselerator.
	\item Kecepatan akselerator dibanding dengan implementasi perangkat lunak.
\end{enumerate}

Ketiga komponen tersebut, akan diuji dan dibahas hasilnya pada bab ini. Akurasi algoritma akan ditinjau terhadap implementasi \ac{DFS}, begitu pula dengan akselerator. Kecepatan yang akan ditinjau adalah kecepatan pada tahap pembelajaran dan kecepatan pada tahap \textit{inference}. Pengujian kecepatan pada kedua tahap ini dilakukan menggunakan modul \textit{timer} yang dapat menghasilkan perbedaan \textit{clock cycles} dari hasil komputasi pada \textit{processor} VeeR EL2. Implementasi timer yang dibangun pada \textit{processor} VeeR EL2 merupakan adopsi dari spesifikasi OpenCores yang tersedia secara terbuka pada \parencite{open2024ptc}.

\begin{enumerate}
	\item Pembelajaran \acl{RL}\\
	      Pengujian ini akan mencoba membandingkan algoritma \ref{alg:rl-qmemo} yang menggunakan perangkat lunak terhadap algoritma \ref{alg:hw-sw-sep} yang menggunakan akselerator perangkat keras. Keduanya akan diuji pada \textit{processor} yang sama yaitu VeeR EL3.
	\item \textit{Inference} \acl{RL}\\
	      Pengujian ini akan mencoba membandingkan proses pemilihan aksi $a_{max}$ menggunakan metode eksploitasi. Perbandingan dilakukan dari perangkat lunak terhadap akselerator perangkat keras dengan instruksi q.max.
\end{enumerate}

Pada perbandingan performa \textit{inference}, hasil perbedaan waktu delay yang dibutuhkan oleh perangkat lunak dan akselerator akan diekstrapolasi untuk kasus \textit{cart pole balancing} dari sub-bab \ref{sec:rl-kontrol}. Tujuan akhirnya adalah agar dapat memperlihatkan dampak penggunaan akselerator pada suatu kasus fisis yang memerlukan kontrol secara \textit{real time}.

Pada pengujian tersebut, lingkungan pengujian permasalahan \textit{cart pole balancing} dilakukan menggunakan OpenAI Gymnasium \parencite{towers2023gymnasium}. Pada lingkungan OpenAI Gymnasium, permasalahan \textit{cart pole balancing} memiliki spesifikasi pada tabel \ref{tab:state-space-cart-pole}.

\begin{table}[h!]
	\caption{Spesifikasi $State$ untuk \textit{Cart Pole Balancing} dari OpenAI Gymnasium}
	\label{tab:state-space-cart-pole}
	\centering
	\begin{tabular}{|c|c|c|}
		\hline
		Observation           & Min                             & Max                           \\
		\hline
		Cart Position         & -4,8                            & 4,8                           \\
		\hline
		Cart Velocity         & $-\infty$                       & $\infty$                      \\
		\hline
		Pole Angle            & $\sim -0,418$ rad $(-24^\circ)$ & $\sim 0,418$ rad $(24^\circ)$ \\
		\hline
		Pole Angular Velocity & $-\infty$                       & $\infty$                      \\
		\hline
	\end{tabular}
\end{table}

Dengan spesifikasi $state$ sesuai dengan tabel \ref{tab:state-space-cart-pole}, permasalahan \textit{cart pole balancing} dari OpenAI Gymnasium memiliki aksi diskrit sebuah gaya $F$ dengan nilai yang fix, dengan pilihan pemberian gaya ke arah kiri atau kanan. Gambar \ref{fig:cartpole-openai} merupakan ilustrasi yang diberikan oleh OpenAI Gymnasium untuk lingkungan permasalahan \textit{cart pole balancing}.

\begin{figure}[h]
	\centering
	\includegraphics[width=1\textwidth]{chapter-3/cart-pole-openai.jpg}
	\caption{Ilustrasi lingkungan permasalahan \textit{cart pole balancing} OpenAI}
	\label{fig:cartpole-openai}
\end{figure}

Tujuan dari lingkungan permasalahan pada gambar \ref{fig:cartpole-openai} adalah agar terbentukan sebuah model agen \ac{RL} yang mampu menentukan arah gaya yang sesuai untuk setiap saat sehingga \textit{cart pole} tidak jatuh. Tentunya, delay dari komputasi akan sangatlah berpengaruh kepada pengambilan keputusan oleh agen \ac{RL}. Pada permasalahan di penelitian ini, aksi diskrit yang digunakan akan dibuat lebih besar, tidak hanya dua kemungkinan. Kemungkinan dari aksi diskrit akan diperpanjang dengan sebuah limit [$-F$, $F$] dengan pencacahan diskrit sehingga aksi menjadi sebanyak 1.000 kemungkinan. Hal ini dilakukan untuk meninjau kemampuan akselerator untuk mengatasi permasalahan \textit{large branching factor} \parencite{amado2018qtable}.

\section{Akurasi Implementasi \acl{RL}}

Terdapat dua metode yang diajukan sebagai \textit{improvement} komputasi \ac{RL} pada penelitian ini: algoritma \ref{alg:rl-qmemo} yang mampu melakukan memoisasi agar memperbaik hasil komputasi \ac{RL} dan akselerator perangkat keras. Hasil uji akurasi terhadap kedua metode ini akan dibahas pada sub subbab selanjutnya.


% TODO: bikin bar chart untuk hasil comparison software dan hardware
\section{Akurasi Implementasi Algoritma \ref{alg:rl-qmemo} pada Perangkat Lunak}

Pengujian akurasi algoritma \ref{alg:rl-qmemo} dilakukan dengan pembandingan hasil akhir yang didapat dari algoritma tersebut dengan implementasi \ac{DFS}. Kedua algoritma dibandingkan hasil jalur tercepatnya dari labirin berdimensi 2 sampai 10 yang dihasilkan dari algoritma Prim pada \ref{alg:prim}. Gambar \ref{fig:akurasi-qmemo} memperlihatkan hasil plot dari variasi dimensi labirin terhadap banyak langkah yang diambil pada labirin untuk menyelesaikannya.

\begin{figure}[h]
	\centering
	\includegraphics[width=1\textwidth]{chapter-4/plot-akurasi-qmemo.png}
	\caption{Hasil pengujian akurasi algoritma \ref{alg:rl-qmemo}}
	\label{fig:akurasi-qmemo}
\end{figure}

Gambar \ref{fig:akurasi-qmemo} menunjukkan bahwa agen \ac{RL} dapat melakukan pembelajaran secara baik karena dapat menghasilkan nilai tempuh yang optimal sesuai dengan hasil dari \ac{DFS}. Selanjutnya, dilakukan perbandingan \textit{cumulative rewards} yang didapatkan dari hasil pembelajaran agen \ac{RL} dari algoritma konvensional \ref{alg:rl-qlearning} dan dengan memoisasi \ref{alg:rl-qmemo}. Konfigurasi yang digunakan untuk pengujian ini dilakukan untuk memperlihatkan efektifitas algoritma \ref{alg:rl-qmemo} dengan menyelesaikan masalah labirin dengan dimensi 20 dan 40. Dimensi tersebut dipilih agar dapat merepresentasikan contoh kasus dalam posibilitas \textit{state} yang banyak. Gambar \ref{fig:memo-result} merupakan hasil plot dari pengujian tersebut.

\begin{figure}[h]
	\centering
	\includegraphics[width=1\textwidth]{chapter-4/memo-result.png}
	\caption{Hasil pengujian algoritma \ref{alg:rl-qmemo}}
	\label{fig:memo-result}
\end{figure}

Hasil pada gambar \ref{fig:memo-result} memperlihatkan bahwa algoritma \ref{alg:rl-qmemo} memiliki konsistensi lebih baik dibanding kepada algoritma konvensional yang digunakan pada algoritma \ref{alg:rl-qlearning}. Namun, algoritma \ref{alg:rl-qmemo} memiliki kekurangan yaitu terkadang lambat untuk mempelajari alternatif $\pi^*$ yang optimal. Sehingga, algoritma ini baik untuk digunakan pada kasus dimana perubahan \textit{cumulative reward} yang sedikit akan berpengaruh secara besar. Contoh dari kasus tersebut adalah kasus \textit{reinforcement learning} yang menggunakan aksi diskrit seperti permasalahan labirin pada penelitian ini.

\section{Akurasi Implementasi pada Akselerator}

Pengujian akurasi akselerator dilakukan menggunakan setup seperti digambarkan pada sub sub-bab \ref{subsec:verilator}. Pengujian akurasi dilakukan untuk seluruh instruksi dengan konfigurasi dan hasil \textit{timing diagram} pada lampiran \ref{appendix:verilator}. Hasil pada lampiran \ref{appendix:verilator} menunjukkan bahwa implementasi akselerator memiliki akurasi 100\% sesuai dengan implementasi dari perangkat lunak.

\section{Utilisasi Sumber Daya Akselerator}

Pada subbab \ref{sec:fpga}, metrik utilisasi sumber daya terdiri atas 3 hal:

\begin{enumerate}
	\item \acf{LUTs}\\
	      \ac{LUTs} adalah blok dasar dari logika yang digunakan dalam \ac{FPGA}. Metrik penggunaan \ac{LUTs} mengindikasikan seberapa banyak sumber daya logika yang digunakan oleh desain pada \ac{FPGA}. Semakin tinggi penggunaan \ac{LUTs}, semakin kompleks dan padat logika yang diimplementasikan.
	\item \textit{Flip-flops}\\
	      \textit{Flip-flops} adalah elemen penyimpanan dasar dalam \ac{FPGA} yang digunakan untuk menyimpan bit data. Penggunaan \textit{flip-flops} mencerminkan kebutuhan desain terhadap penyimpanan data sementara dan kemampuan untuk menangani operasi sekuensial.
	\item \textit{Block \ac{RAM}}\\
	      \textit{Block \ac{RAM}} adalah memori terintegrasi dalam \ac{FPGA} yang digunakan untuk menyimpan data dalam jumlah besar. Metrik penggunaan \textit{block \ac{RAM}} memberikan gambaran tentang kebutuhan desain terhadap memori, termasuk berapa banyak data yang perlu disimpan dan diakses secara cepat selama operasi.
\end{enumerate}

Berikut, pada Gambar \ref{fig:plot-resource}, merupakan hasil utilisasi sumber daya akselerator dari sintesis desain \textit{processor} VeeR EL2 tanpa dan dengan implementasi akselerator.

\begin{figure}[h]
	\centering
	\includegraphics[width=1\textwidth]{chapter-4/plot-resource.png}
	\caption{Plot hasil penggunaan sumber daya pada \ac{FPGA}}
	\label{fig:plot-resource}
\end{figure}

Dapat diperhatikan pada Gambar \ref{fig:plot-resource}, perubahan banyaknya penggunaan \ac{LUTs}, \textit{flip-flops}, dan \textit{block \ac{RAM}} itu relatif sedikit. Lebih jelasnya lagi, Gambar \ref{fig:plot-resource-increase} mendeskripsikan tentang kenaikan persentase penggunaan sumber daya tersebut beserta dengan banyaknya sumber daya yang masih tersisa pada \ac{FPGA}.

\begin{figure}[h]
	\centering
	\includegraphics[width=1\textwidth]{chapter-4/plot-resource-increase.png}
	\caption{Plot penambahan penggunaan sumber daya pada \ac{FPGA}}
	\label{fig:plot-resource-increase}
\end{figure}

Seluruh peningkatan sumber daya masih berada dibawah 1.5\% dengan sisa sumber daya tersedia yang masih sangatlah banyak, berarti implementasi akselerator perangkat keras berhasil dibangun secara efisien. Selanjutnya, bila dibandingkan dengan implementasi akselerator-akselerator penelitian sebelumnya yang terdapat pada subbab \ref{sec:accelerator-researches} maka didapatkan data sesuai dengan Tabel \ref{tab:comparison-utilization}.


\begin{table}[h]
	\centering
	\caption{Perbandingan utilisasi sumber daya dengan akselerator riset sebelumnya}
	\label{tab:comparison-utilization}
	\renewcommand{\arraystretch}{1.2}
	\setlength{\tabcolsep}{3pt}
	\begin{tabularx}{\textwidth}{|p{20mm}|X|X|X|X|X|X|X|}
		\hline
		\textbf{Reference}            & \multicolumn{2}{c|}{Spanò et al. \parencite{spano2019efficient}} & Da Silva et al. \parencite{dasilva2019parallel} & Y. Meng et al. \parencite{meng2020generic} & \multicolumn{2}{c|}{Sutisna et al. \parencite{sutisna2023faraneq}} & Proposed                            \\ \hline
		\textbf{Design Level}         & \multicolumn{4}{c|}{Standalone Core}                             & \multicolumn{3}{c|}{System on Chip}                                                                                                                                                                     \\ \hline
		\textbf{Action Policy}        & \multicolumn{2}{c|}{NA}                                          & random                                          & $\epsilon$-greedy                          & \multicolumn{2}{c|}{decreasing-$\epsilon$}                         & $\epsilon$-greedy                   \\ \hline
		\textbf{Number of Agents (G)} & single                                                           & single                                          & single                                     & double                                                             & single            & single & single \\ \hline
		\textbf{Bit Width}            & 16                                                               & 32                                              & 30                                         & 16                                                                 & 16                & 32     & 32     \\ \hline
		\textbf{LUT}                  & 333                                                              & 682                                             & 77574                                      & 172                                                                & 2490              & 2062   & 759    \\ \hline
		\textbf{Registers}            & 258                                                              & 606                                             & 13175                                      & NA                                                                 & 2348              & 2179   & 1259   \\ \hline
		\textbf{BRAM}                 & NA                                                               & NA                                              & 266                                        & NA                                                                 & 22                & 20     & 1      \\ \hline
	\end{tabularx}
\end{table}

Pada tabel \ref{tab:comparison-utilization}, dapat diperhatikan bahwa implementasi akselerator yang dibangun pada riset ini merupakan akselerator yang terefisien kedua pada ukuran 32-bit. Hal ini, menunjukkan kemungkinan penggunaan akselerator pada tahap produksi karena dapat bersaing dengan \textit{state of the art} dari penelitian terkini.

\section{Performa Kecepatan Akselerator}

Sebagaimana disebutkan pada subbab \ref{sec:pengujian-performa-akselerator}, terdapat dua pengujian performa kecepatan yang penting untuk diujikan yaitu performa kecepatan akselerator dalam proses pembelajaran \ac{RL} dan dalam proses \textit{inference}. Kedua proses ini digunakan di waktu yang berbeda dan memiliki signifikansi aplikasi yang berbeda juga. Selengkapnya, hasil data dan pembahasan yang didapat untuk uji performa akselerator ini akan dibahas pada sub subbab \ref{subsec:performance-learning} dan \ref{subsec:performance-inference}.

\subsection{Performa Pembelajaran \acl{RL}}
\label{subsec:performance-learning}

Untuk menguji performa pembelajaran \ac{RL} pada akselerator, dilakukan pengujian kecepatan untuk melakukan pembelajaran masalah labirin dengan besar 5 x 5. Berikut, pada Gambar \ref{fig:5x5}, merupakan hasil dari hasil pengujian.

\begin{figure}[h]
	\centering
	\includegraphics[width=1\textwidth]{chapter-4/5x5.png}
	\caption{Hasil pengujian akselerator terhadap perangkat lunak pada besar labirin 5 x 5}
	\label{fig:5x5}
\end{figure}

Gambar \ref{fig:5x5} merupakan plot grafik antara \textit{cumulative reward} terhadap waktu yang diperlukan untuk melakukan pembelajaran. Dapat dilihat, bahwa penggunaan akselerator mempercepat pembelajaran tanpa mengubah nilai \textit{cumulative reward} sama sekali. Pengujian pada gambar \ref{fig:5x5}, dilakukan dengan 1000 jumlah episode pembelajaran. Agar dapat melihat tren dari waktu pembelajaran terhadap episode, dibuatlah grafik yang diplot pada gambar \ref{fig:episode_vs_actime}.

\begin{figure}[htbp]
	\centering
	\includegraphics[width=1\textwidth]{chapter-4/episode_vs_actime.png}
	\caption{Hasil pengujian akselerator pada berbagai macam episode maksimal}
	\label{fig:episode_vs_actime}
\end{figure}

Pada gambar \ref{fig:episode_vs_actime}, diplot tren waktu yang diambil terhadap variasi episode. Kedua implementasi, baik pada perangkat lunak maupun pada akselerator perangkat keras, memiliki tren linear dalam peningkatan waktu yang diperlukan terhadap episode. Namun, implementasi perangkat keras memiliki gradien yang lebih kecil dari implementasi perangkat lunak. Setelah dilakukan regresi, didapat bahwa masing-masing garis memiliki persamaan sebagai berikut.

\begin{enumerate}
	\item Perangkat lunak: $y = 5494.05x + 7364.74$
	\item Akselerator perangkat keras: $y = 3816.97x - 4681.35$
\end{enumerate}

Dengan demikian, dapat disimpulkan bahwa meskipun kompleksitas teoritis dari perangkat lunak dan akselerator perangkat keras sama, $O(n)$, tetapi implementasi akselerator perangkat keras akan menjadi semakin efisien bila dibandingkan kepada implementasi perangkat lunak pada episode yang semakin tinggi.

\subsection{Performa \textit{Inference} \acl{RL}}
\label{subsec:performance-inference}

Pada pengujian performa akselerator untuk proses \textit{inference}, dilakukan pengukuran waktu yang diperlukan untuk melakukan metode $\epsilon$-greedy. Pada implementasi akselerator perangkat keras, digunakan instruksi q.max sedangkan pada implementasi perangkat lunak dilakukan pembandingan nilai maksimal menggunakan kondisional dan pengulangan. Berikut merupakan hasil yang didapatkan dari uji pada permasalahan labirin.

\begin{enumerate}
	\item Perangkat lunak: 393 \textit{clock cycles}
	\item Akselerator perangkat keras: 13 \textit{clock cycles}
\end{enumerate}

Kecepatan yang didapatkan dari akselerator mencapai sekitar 30.23 kali lebih cepat daripada perangkat lunak. Kecepatan ini kemudian dicoba dengan variasi $a_{max}$ yang dilakukan agar dapat melihat sebaik apa penggunaan akselerator pada \textit{large branching factor}. Gambar \ref{fig:speed-action} merupakan plot dari performa akselerator dengan variasi $a_{max}$.

\begin{figure}[htbp]
	\centering
	\includegraphics[width=1\textwidth]{chapter-4/plot-speed-action.png}
	\caption{Hasil pengujian kecepatan akselerator dengan variasi $a_{max}$}
	\label{fig:speed-action}
\end{figure}

Gambar \ref{fig:speed-action} memiliki sifat yang cukup mirip dengan gambar \ref{fig:episode_vs_actime}, yaitu sifat linear dari kedua profil kecepatan. Namun, perbedaan gradien yang dimiliki oleh \ref{fig:speed-action} jauh lebih besar lagi. Berikut merupakan detail dari masing-masing persamaan garis.

\begin{enumerate}
	\item Perangkat lunak: $y = 120.36x - 12.55$
	\item Akselerator perangkat keras: $y = 3x - 3.19$
\end{enumerate}

Apabila digunakan $a_{max}$ yang bernilai 1000, maka banyaknya \textit{clock cycles} yang diperlukan untuk menyelesaikan $\epsilon$-greedy adalah 120342 pada perangkat lunak dan 2992 pada akselerator perangkat keras. \textit{Processor} VeeR EL2 yang digunakan pada eksperimen ini memiliki kecepatan 12.5 MHz. Sehingga waktu \textit{delay} adalah 9 milisekon untuk perangkat lunak dan 0.2 milisekon untuk akselerator perangkat lunak. Waktu \textit{delay} tersebut digunakan untuk mensimulasikan implikasi \textit{real time} pada permasalahan \textit{cart pole balancing}.

Hasil percobaan menggunakan permasalahan \textit{cart pole balancing} memperlihatkan bahwa dengan delay 9 milisekon dari perangkat lunak, \textit{cart pole} hanya bisa bertahan 4 detik sebelum akhirnya terjatuh. Sedangkan, pada kasus delay 0.2 milisekon dari akselerator perangkat keras \textit{cart pole} berhasil mempertahankan posisinya setelah selama 1 menit.


% Bab ini akan menjelaskan proses implementasi dari rancangan solusi yang telah dikaji pada Bab III. Setelah pembahasan terkait implementasi, akan dilanjutkan dengan pemaparan hasil uji terkait implementasi yang telah dibuat.

% \section{Lingkungan}

Sistem kontrol adaptif akan diimplementasikan di lingkungan komputer lokal. Berikut adalah lingkungan perangkat keras dan perangkat lunak secara terperinci.

\subsection{Lingkungan Perangkat Keras}
\blindtext

\subsection{Lingkungan Perangkat Lunak}
\blindtext
% 
% \section{Implementasi}

Bagian ini akan menjelaskan tentang implementasi sistem kontrol adaptif secara terperinci.

\subsection{Batasan Implementasi}
\blindtext

\subsection{Kakas yang Digunakan}
\blindtext

\subsection{Komponen \textit{Metrics Fetcher}}
\blindtext

\subsection{Komponen \textit{Predictor}}
\blindtext

\subsection{Komponen \textit{Rule Manager}}
\blindtext

\subsection{Komponen \textit{Resource Controller}}
\blindtext

\subsection{Komponen \textit{Adaptive Control}}
\blindtext

\subsection{Tampilan Implementasi}
\blindtext
% 
% \section{Pengujian}

Bagian ini akan menjelaskan beberapa skenario yang dilakukan untuk menguji sistem kontrol adaptif.

\subsection{Pengujian X}
\blindtext

\subsubsection{Tujuan Pengujian}
\blindtext

\subsubsection{Skenario Pengujian}
\blindtext

\subsubsection{Hasil Pengujian dan Analisis}
\blindtext
% 
% \section{Perbandingan}

Bagian ini akan menjelaskan pengujian dampak dari \textit{autoscaler} dengan kontrol fleksibel. Pengujian akan dilakukan dengan membandingkan \textit{autoscaler} yang dibuat dengan sistem kontrol fleksibel dengan \textit{autoscaler} sederhana yang memakai \textit{treshold} serta tanpa \textit{autoscaler}. 

\subsection{Perbandingan Performa}

\subsection{Perbandingan Efisiensi}
