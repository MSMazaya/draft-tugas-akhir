\chapter{Analisis Masalah dan Rancangan Solusi}

Tujuan utama penulisan bab ini adalah untuk menguraikan rencana penyelesaian masalah tugas akhir yang akan dieksekusi secara utuh pada saat pelaksanaan Tugas Akhir II. Bab ini merupakan bab penutup Laporan Tugas Akhir I yang dapat dipandang sebagai bab yang akan menjembatani perpindahan ke proses pelaksanaan Tugas Akhir II. Pengembangan lebih lanjut dari bab ini dapat menjadi bagian dari bab Deskripsi Solusi pada Laporan Tugas Akhir.

\section{Analisis Masalah}

% todo analisis persoalan

Untuk membuat kontrol adaptif untuk menangani masalah efisiensi sumber daya berdasarkan kinerja aplikasi serta kondisi jumlah permintaan dalam satu satuan waktu dalam jangka waktu tertentu, akan diajukan dua buah pendekatan solusi. Berikut adalah analisis dari dua pendekatan yang diajukan pada tugas akhir ini.

% todo pemetaan masalah ke solusi

% \subsection{Vertical Pod Autoscaler dari Kubernetes}

% Kubernetes sendiri saat ini sedang mengembangkan Vertical Pod Autoscaler (VPA), \parencite{vpa}. Fitur ini sudah bisa dipakai meskipun masih dalam tahap pengembangan. Namun, jika menggunakan pendekatan ini, terdapat beberapa \textit{drawback}. Pertama, VPA perlu melakukan \textit{restart} terhadap \textit{pod} yang ingin dibesarkan sedangkan \textit{Elastic Search} menggunakan sistem sharding, apabila mematikan salah satu \textit{Node Elastic Search} maka hal tersebut akan menyebabkan \textit{Elastic Search} perlu melakukan \textit{balancing shard data} setiap kali \textit{autoscale} yang tentunya akan memakan ketersediaan dan sumber daya. Kedua, VPA menggunakan \textit{metrics} yang didapatkan dari Kubernetes bukan dari Elastic Search, hal ini akan menyebabkan kurangnya akurasi dan atau tidak tercapainya tujuan untuk membebaskan memori yang dipakai namun dampaknya tidak signifikan terhadap kinerja \textit{Elastic Search} itu sendiri. Oleh karena itu, rancangan solusi dengan hal ini dirasa kurang cocok.

\section{Rancangan Solusi}

\section{Sistem Kontrol Adaptif}
\label{sec:sistemkontroladaptif}

Sistem Kontrol Adaptif akan memanfaatkan \textit{metrics} yang didapat dari \textit{Elastic Search} secara periodik. Data tersebut akan dijadikan faktor dalam membuat keputusan untuk memperbesar atau memperkecil limit memori \textit{Elastic Search} tersebut. Sistem tersebut secara umum akan dibagi menjadi dua komponen, yaitu \textit{Metrics Collector} dan \textit{Memory Controller}.

Seperti yang sudah dijelaskan pada bagian sebelumnya, \ref{sec:sistemkontroladaptif}, Sistem Kontrol Adaptif akan disusun atas dua komponen, yaitu sebagai berikut.
\subsection{Komponen \textit{Metrics Collector}}
\label{sec:metricscollector}

Komponen ini bertugas untuk menarik data \textit{metrics} dari \textit{Elastic Search} menggunakan HTTP Client dan Database Connector Client. Data tersebut lalu akan disimpan ke sebuah database relasional yang dapat digunakan sebagai data historis. Sehingga, kedepannya dapat dilakukan analisis diagnostik dan analisis prediktif.

\subsection{Komponen \textit{Decision Maker}}
Komponen \textit{Decision Maker} (DM) akan melakukan prediksi dan membuat model berdasarkan data historis yang sudah dikumpulkan.
Komponen ini akan menggunakan data yang dikumpulkan oleh komponen Metrics Collector, \ref{sec:metricscollector}.
Kakas yang akan dipakai, diantaranya HTTP Client, Database Connector Client, dan \textit{Library Machine Learning} atau \textit{Statistics}, jika diperlukan.

\subsection{Komponen \textit{Resource Controller}}

Komponen ini bertugas untuk memperbesar, membiarkan atau memperkecil resource \textit{Elastic Search}. Kakas yang akan digunakan oleh komponen ini adalah Kubernetes Client Library dan HTTP Client.

% \subsection{\textit{Data Gathering: Greedy, Trial and Error}}

