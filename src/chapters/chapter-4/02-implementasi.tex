\section{Implementasi}

Bagian ini akan menjelaskan tentang implementasi sistem kontrol adaptif secara terperinci.

\subsection{Batasan Implementasi}
Berikut adalah batasan yang ditetapkan dalam melakukan implementasi sistem kontrol adaptif.
\begin{enumerate}
    \item Masih menerapkan sistem \textit{single node} dan \textit{single pod}.
    \item Tidak memperhatikan optimalisasi dari model ARIMA.
    \item Hanya dapat menangani metrik yang sudah ditentukan, yaitu \textit{throughput} dari setiap operasi \textit{Elastic Search} serta utilisasi prosesor hingga memori.
    \item Tidak mempertimbangkan besarnya model seiring bertambahnya data.
    \item Komponen \textit{Metrics Fetcher} berjalan di proses lain dan diimplementasikan dalam \textit{script} yang berbeda dikarenakan bahasa Python memiliki kekurangan dalam penanganan \textit{multithreading}.
    \item Pertukaran data antara komponen \textit{Metrics Fetcher} dan \textit{Predictor} melalui stream file.
\end{enumerate}

\subsection{Kakas yang Digunakan}
Dalam melakukan implementasi ini diperlukan beberapa kakas, diantaranya adalah sebagai berikut.
\begin{enumerate}
    \item \textit{Docker}, \textit{Docker Desktop} dan \textit{Docker Desktop Kubernetes} untuk dipakai sebagai \textit{containerization} dan \textit{cluster} kubernetes lokal.
    \item Pandas dan Numpy untuk keperluan \textit{data processing} serta bentuk data untuk dikirimkan ke komponen lain serta model prediksi ARIMA.
    \item \textit{Kubernetes Python Client} untuk mengontrol \textit{cluster} kubernetes melalui kode Python.
    \item \textit{Pickle} untuk menyimpan model ARIMA sehingga persisten meskipun sistem di-\textit{restart}.
    \item \textit{Statsmodels} dan \textit{pmdarima} untuk membangun model ARIMA serta melakukan otomasi pencarian orde atau lebih dikenal sebagai Auto-ARIMA.
\end{enumerate}

\subsection{Persiapan \textit{Pods Elastic Search}}

Sebelum melakukan implementasi, diperlukan untuk menyalakan \textit{Pods Elastic Search}. Konfigurasi ini dilakukan dengan cara membuat \textit{file deployment}, contoh dapat dilihat pada gambar \ref{fig:spek-deployment}.

\begin{figure}[h]
    \centering
    \includegraphics[width=0.75\textwidth]{chapter-4/spek-deployment.png}
    \caption{Konfigurasi \textit{Pods Elastic Search}}
    \label{fig:spek-deployment}
\end{figure}

\subsection{Komponen \textit{Metrics Fetcher}}
\textbf{\textit{Metrics Fetcher}} merupakan komponen yang berbeda dibanding komponen lainnya, karena komponen ini berjalan pada \textit{script} serta proses yang berbeda. Seperti yang sudah dijelaskan sebelumnya, komponen ini akan menembak permintaan HTTP pada \textit{Node Stats API} (dokumentasi dapat dilihat pada tautan \url{https://www.elastic.co/guide/en/elasticsearch/reference/current/cluster-nodes-stats.html}) yang telah disediakan \textit{Elastic Search}. Komponen ini akan melakukan transformasi bentuk data menjadi lebih sederhana dan sesuai kebutuhan komponen lainnya. Komponen dibuat pada \textit{script} yang berbeda dan berjalan secara paralel pada proses lain dikarenakan bahasa Python memiliki kekurangan dalam penanganan \textit{multithreading}. Komponen ini akan mengirimkan data yang sudah diolah ke komponen \textbf{\textit{Predictor}} melalui \textit{stream file}. Pendekatan ini dipilih karena sederhana dan mudah diimplementasikan. Khusus komponen ini, struktur kodenya tidak memakai sistem kelas dan hanya terdapat sebuah fungsi dan beberapa baris perintah untuk melakukan pemanggilan API, transformasi data dan pengiriman ke \textit{stream file}.
\subsection{Pengujian Komponen \textit{Predictor}}

Pada bagian ini akan dijelaskan tentang tujuan, skenario, hasil, dan analisis dari pengujian komponen \textbf{\textit{Predictor}}.

\subsubsection{Tujuan Pengujian}

Tujuan pengujian ini memastikan komponen \textbf{\textit{Predictor}} dapat berjalan dengan baik dan menghasilkan data yang sesuai dengan ekspektasi.

\subsubsection{Skenario Pengujian}

Pengujian terhadap komponen \textbf{\textit{Predictor}} dilakukan dengan membandingkan hasil prediksi dengan aktual untuk skenario sebagai berikut.
\begin{enumerate}
    \item \textit{Elastic Search} sedang \textit{idle}.
    \item \textit{Elastic Search} sedang digunakan untuk melakukan operasi penambahan data.
    \item \textit{Elastic Search} sedang digunakan untuk melakukan operasi pencarian data.
\end{enumerate}

\subsubsection{Hasil Pengujian dan Analisis}

% \begin{figure}[h]
%     \centering
%     \includegraphics[width=0.8\textwidth]{chapter-4/mf-3.png}
%     \caption{Hasil Pengujian Komponen \textit{Metrics Fetcher} Skenario 3}
%     \label{fig:mf-3}
% \end{figure}

Pengujian komponen \textbf{\textit{Predictor}} menghasilkan angka yang cukup baik sehingga hasil prediksinya bisa dianggap merepresentasikan kondisi aktual.
\subsection{Komponen \textit{Rule Manager}}
Komponen \textbf{\textit{Rule Manager}} berfungsi untuk melakukan parsing terhadap file \textit{rule} yang telah diisi oleh pengguna serta menjadi aggregator untuk melakukan pengecekan \textit{rule} yang berlangsung serta memberi informasi data prediksi kapan saja yang dibutuhkan untuk melakukan pengecekan. Parsing komponen ini menggunakan format csv dan kondisi diekspresikan dengan sintaks python. Komponen akan mengonstruksi objek \textbf{\textit{Rule}} yang akan digunakan oleh komponen \textbf{\textit{Flexible Control}}. Agar terbayang, contoh dari \textit{file rule} dapat dilihat pada lampiran XXX. Spesifikasi dari kedua kelas tersebut dapat dilihat pada gambar \ref{fig:rule-spek}.

% TODO CONTOH RULE, masukin ke lampiran, trus tag kesini.
% TODO PRINSIP RULE, POSSIBILITIES, DAN BISA NGAPAIN AJA
% TODO CARA MENYUSUN FILE RULE

Sebuah \textit{rule} memiliki fungsi sebagai berikut.
\begin{enumerate}
    \item Memiliki sebuah kondisi yang akan dievaluasi dengan data prediksi pada waktu prediksi yang diinginkan. Contoh: kondisi \textit{throughput} untuk operasi X untuk 1 menit kedepan dan 5 menit kedepan lebih dari 1s, maka tingkatkan prosesor sebanyak 500m.
    \item Memiliki jumlah serta target kategori untuk diubah, dalam kasus ini pilihannya memori atau prosesor.
    \item Satuan untuk perubahan memori adalah dalam \textit{Mebibyte} atau MiB. Sedangkan untuk prosesor dalam satuan mili atau m.
    \item Sebuah \textit{rule} memiliki periode pengecekan sehingga tidak akan dicek secara terus menerus yang menyebabkan perubahan alokasi sumber daya terlalu cepat. Periode pengecekan dibuat dalam satuan sekon.
\end{enumerate}

\begin{figure}[h]
    \centering
    \includegraphics[width=0.8\textwidth]{chapter-4/rule.png}
    \caption{Spesifikasi Kelas Penyusun Komponen \textit{Rule Manager}}
    \label{fig:rule-spek}
\end{figure}
\subsection{Komponen \textit{Resource Controller}}

Komponen ini terdiri dari sebuah kelas. Seperti namanya, kelas ini berfungsi untuk menggunakan \textit{Kubernetes Client API} untuk mengubah alokasi sumber daya. Kelas ini diimplementasikan dengan sistem antrian, sehingga jika sejumlah rule aktif secara bersamaan, maka akan dijalankan secara berurutan. Terdapat sebuah fungsi \textit{tick} yang akan berfungsi untuk mengeksekusi antrian. Spesifikasi kelas ini dapat dilihat pada gambar \ref{fig:rc-spek}.

\begin{figure}[h]
    \centering
    \includegraphics[width=0.4\textwidth]{chapter-4/rc.png}
    \caption{Spesifikasi Kelas Penyusun Komponen \textit{Resource Controller}}
    \label{fig:rc-spek}
\end{figure}

% TODO CONTOH SISTEM ANTRIAN
\input{chapters/chapter-4/implementasi/05-flexible-control}
\input{chapters/chapter-4/implementasi/06-konfigurasi}