\subsection{Eksperimen \textit{In-place Resource Resize} untuk \textit{pods Kubernetes}}

Pada bagian sebelumnya, diperlukan \textit{resizing} sumber daya yang dialokasikan untuk \textit{pods} yang sedang berjalan tanpa melakukan \textit{restart}. Untuk hal tersebut, perlu dilakukan eksperimen, berikut adalah rincian dari eksperimen yang telah dilakukan.

\subsubsection{Pendahuluan}

\textit{In-place Resource Resize} adalah fitur untuk mengubah ukuran sumber daya CPU dan memori yang dialokasikan untuk kontainer pada pod yang sedang berjalan tanpa harus me-\textit{restart} pod atau kontainernya. Sebuah \textit{node} Kubernetes mengalokasikan sumber daya untuk sebuah pod berdasarkan permintaannya, dan membatasi penggunaan sumber daya pod berdasarkan batasan yang ditentukan dalam kontainer-kontainer pod tersebut. Fitur ini baru hadir pada versi Kubernetes 1.27.0, dan, pada saat tugas akhir ini dikerjakan, masih dalam tahap \textit{alpha testing} dan pengembangan. Berdasarkan \parencite{kubeinplaceupdate2}, dokumentasi resmi dipublikasikan pada 30 Maret 2023 7:59 PM PST. Sedangkan berdasarkan \parencite{kubeinplaceupdate}, publikasi \textit{alpha testing} semenjak 13 Mei 2023.

\subsubsection{Pengerjaan Eksperimen}
Dalam melakukan eksperimen ini, dilakukan beberapa tahap sebagai berikut.

\begin{enumerate}
    \item Memastikan versi Kubernetes yang digunakan adalah versi 1.27.0 atau lebih baru pada \textit{client} dan \textit{server}.
    
    Pada saat itu, Kubernetes lokal yang dipakai adalah \textit{Docker Desktop Kubernetes} yang membatasi versi Kubernetes pada versi 1.25.9. Sehingga, dilakukan pengubahan server dengan menggunakan Minikube. Sayangnya, versi maksimal yang bisa dipakai oleh Minikube adalah Kubernetes versi 1.27.0-rc0, versi 1.27 yang paling pertama atau \textit{release candidate 0}. Untuk mengecek bisa dilihat pada \url{https://github.com/kubernetes/minikube/releases/tag/v1.30.0}. Dicoba juga untuk dipaksa menggunakan versi 1.27.1 maupun 1.27.3, namun hal tersebut gagal untuk dilakukan. Sehingga, untuk eksperimen ini, digunakan Kubernetes versi 1.27.0-rc0 melalui Minikube.

    \item Membuat \textit{deployment} yang akan digunakan untuk eksperimen.
    
    Konfigurasi \textit{deployment} berubah, karena untuk menggunakan fitur ini, tidak bisa membuat \textit{pods} dengan menggunakan tipe \textit{deployment} melainkan harus langsung membuat dengan tipe \textit{pod}. Terdapat juga beberapa konfigurasi baru yang perlu diatur.

    \item Mengeksekusi perintah untuk melakukan \textit{in-place resource resize}.
    
    Hal ini bisa dilakukan dengan mengeksekusi perintah \textit{patch pod}. Perintah ini bisa dilihat pada dokumentasi: \url{https://kubernetes.io/docs/tasks/configure-pod-container/resize-container-resources/}.

    Saat hal ini dijalankan terdapat pesan eror yang mengatakan bahwa fitur \textit{patch} tersebut hanya bisa dilakukan selain resource. Padahal, inisiasi eksperimen sudah disesuaikan dengan \textit{requirement} yang disebutkan pada dokumentasi.

    \item Mengecek detail informasi \textit{pods}
    
    Seharusnya, ketika mengecek detail informasi \textit{pods} akan terlihat bahwa resource yang digunakan sudah berubah dan terdapat informasi-informasi baru yang hanya muncul pada versi terbaru Kubernetes. Namun, hal ini tidak terjadi. Resource yang digunakan masih sama dengan sebelumnya. Dan hasil detail informasi sebuah \textit{pods} tidak memperlihatkan detail informasi terbaru yang sesuai dengan contoh pada dokumentasi.
\end{enumerate}

\subsubsection{Hasil Eksperimen}

Berdasarkan hasil eksperimen tersebut, dapat disimpulkan bahwa.

\begin{enumerate}
    \item Fitur \textit{in-place resource resize} belum bisa digunakan pada Kubernetes versi 1.27.0-rc0.
    \item Melakukan \textit{resource resize} tanpa melakukan \textit{restart} mungkin bisa dilakukan pada masa yang akan datang.
    \item Eksperimen ini bisa dicoba lagi pada masa yang akan datang. Mengingat tools kubernetes lokal masih belum bisa menggunakan versi \textit{alpha} terbaru. Dan, pada saat ini, fitur ini masih dalam tahap \textit{alpha testing} dan pengembangan.
\end{enumerate}

Sehingga, untuk saat ini, eksperimen ini tidak bisa dilakukan lebih jauh lagi. Dan oleh karena itu, sistem kontrol adaptif yang dibuat masih memakai sistem \textit{Rolling Update} untuk mengubah sumber daya alokasi. Namun, kedepannya, jika ingin diteruskan, besar kemungkinan hal ini bisa dilakukan karena dari Kubernetes sendiri sedang mengembangkan fitur tersebut.