\section{Lingkungan}

Sistem kontrol adaptif akan diimplementasikan di lingkungan komputer lokal. Berikut adalah lingkungan perangkat keras dan perangkat lunak secara terperinci.

Implementasi sistem tugas akhir dilakukan dengan mengimplementasikan diatas beberapa kakas pada bahasa \textit{python}. Implementasi dikembangkan di luar \textit{kubernetes cluster} dan mengakses Kubernetes beserta \textit{pods}-nya melalui \textit{Kubernetes Client Library} serta mengekspos port melalui \textit{service} yang disediakan Kubernetes. 

Sistem kontrol adaptif akan berjalan di \textit{cluster} Kubernetes lokal. Penelitian ini juga hanya dibataskan pada \textit{single node} dan \textit{single pods}. Adapun spesifikasi dari komputer yang dipakai untuk pengembangan adalah sebagai berikut.
\begin{enumerate}
    \item \textbf{Perangkat Keras}
    
        \begin{enumerate}
            \item CPU: \textit{Apple M1 Chip}
            \item RAM: 16 GB
        \end{enumerate}
    
    \item \textbf{Perangkat Lunak}
        
        \begin{enumerate}
            \item Platform dan Sistem Operasi: Darwin AMD64, MacOS Monterey 12.6
            \item \textit{Containerization}: Docker
            \item \textit{Kubernetes Cluster}:
                \begin{enumerate}
                    \item Kubernetes Client v1.27.1-eks-2f008fe
                    \item Kubernetes Docker Desktop: Kubernetes v1.25.9
                \end{enumerate}
            \item Bahasa: Python 3.9.12
            \item Dependensi Lain:
                \begin{enumerate}
                    \item \textit{Kubernetes Client Library}
                    \item \textit{Pandas, numpy, statsmodels dan pmdarima}
                    \item \textit{Pickle}
                \end{enumerate}
        \end{enumerate}
\end{enumerate}