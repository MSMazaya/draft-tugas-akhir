\chapter*{Kata Pengantar}
\addcontentsline{toc}{chapter}{KATA PENGANTAR}

Puji syukur kepada Tuhan Yang Maha Esa karena atas berkat-Nya, telah diselesaikan tugas akhir dengan judul "\thetitle".
Atas kontribusi baik secara langsung maupun tidak langsung, penulis ingin
mengucapkan terima kasih kepada:

\begin{enumerate}
	\item Bapak Dr. Ir. Eko Mursito Budi, M.T. dan Bapak Dr. Eng. Infall Syafalni, S.T., M.Sc. selaku dosen pembimbing tugas akhir yang telah membimbing, memberikan arahan, dan mendukung penulisan laporan tugas akhir ini.
	\item Bapak Nana Sutisna, S.T., M.T., Ph.D. dan Kak Yahwista Salomo, S.T. yang telah memberikan bimbingan dan bantuan arahan mengenai metodologi penelitian yang akan dilakukan dan dicantumpan pada laporan tugas akhir.
	\item Orang tua dan keluarga Penulis yang selalu memberikan dukungan penuh dan menyediakan kebutuan serta doa sehingga Penulis dapat menyelesaikan penulisan laporan tugas akhir.
	\item Seluruh Dosen dan Staff Program Studi Teknik Fisika Institut Teknologi Bandung yang telah memberikan banyak bantuan agar dapat tertulisnya laporan tugas akhir.
	\item Prudensia Fairuz Zhafirah yang telah memberikan dukungan moral dan menemani penulisan laporan tugas akhir ini.
	\item Teman-teman Teknik Fisika 2020 yang telah memberikan dukungan moral untuk menyelesaikan penulisan laporan tugas akhir ini.
	\item Semua pihak yang tidak dapat disebutkan satu per satu. Semoga laporan tugas akhir ini dapat memberikan gambaran baik tentang tugas akhir yang akan dilakukan.
\end{enumerate}

Akhir kata, penulis mengucapkan terima kasih kepada semua pihak yang telah terlibat dalam pengerjaan tugas akhir ini. Penulis juga ingin menyampaikan mohon maaf apabila terdapat kesalahan maupun kekurangan dalam laporan tugas akhir ini. Penulis berharap semoga tugas akhir ini dapat bermanfaat bagi pembaca dan riset-riset kedepannya.

\begin{flushright}
	\vspace{0.5cm}
	Bandung, \tanggalpengesahan
	
	
	\vspace{1.5cm}
	
	Muhammad Sulthan Mazaya
\end{flushright}
