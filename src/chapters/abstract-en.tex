\clearpage
\chapter*{\textit{ABSTRACT}}
\addcontentsline{toc}{chapter}{ABSTRACT}

\begin{center}
	\begin{singlespace}
		\center
		\large\bfseries\MakeUppercase{\textit{DEVELOPMENT OF RISC-V BASED HARDWARE ACCELERATOR FOR REINFORCEMENT LEARNING}}
		
		\normalfont\normalsize
		
		By\\
		\bfseries{\textit{Muhammad Sulthan Mazaya \hspace{5mm} NIM: 13320028}}
		
		\vspace{5mm}
		\large\bfseries{\textit{(Engineering Physics Study Program)}}
		\vspace{5mm}
	\end{singlespace}
\end{center}


\begin{singlespace}
	\small
	\textit{Reinforcement Learning (RL) is one of the popular frameworks for developing
		autonomous agents. RL serves as an alternative modeling solution for problems in
		systems that are too complex to be mathematically or algorithmically modeled. As
		a result, RL is widely employed in domains such as robotics and autonomous driver
		agents. However, despite being a suitable alternative for many problems, RL is
		often challenging to use due to hardware resource limitations. This is primarily
		because RL typically demands significant hardware resources. This can be a
		hindrance when deploying RL models on computers with limited resources, such as
		Internet of Things (IoT) devices or edge computing platforms. Therefore, in this
		research, a akselerator perangkat keras design is proposed to reduce the
		computational power required for RL algorithm computation. This akselerator
		perangkat keras design will be implemented on a Field Programmable Gate Array
		with an RISC-V, an open instruction set architecture, architecture in the form of a
		co-processor. The results achieved in this final project consist of hardware and
		software configurations, along with the design of the software and hardware that
		will be implemented.
	}
	
	\textit{Keywords: reinforcement learning, co-processor, field programmable gate array, RISC-V}
\end{singlespace}
\clearpage

\clearpage
