\section{Analisis Persoalan}

Kontrol fleksibel merupakan salah satu teknologi yang bermanfaat untuk membantu pengelola infrastruktur dalam melakukan \textit{autoscale} terutama dengan teknologi \textit{container orchestration} seperti Kubernetes. Dengan menggunakan kontrol fleksibel, pengguna dapat mengatur dan mengontrol (\textit{scaling}) sumber daya dari sekumpulan \textit{pods} dengan otomatis sesuai \textit{rule} yang ditetapkan oleh pengelola. Beberapa cara yang paling populer saat ini adalah \textit{horizontal autoscaling} dan \textit{vertical autoscaling} berdasarkan metrik sistem seperti utilisasi prosesor, memori serta beban permintaan dalam satuan waktu. Perkembangan dari metode \textit{autoscaling} telah menyediakan pengguna berbagai opsi untuk mengotomasi alokasi sumber daya. Namun, dari semua metode yang sekarang sudah ada, terdapat beberapa kekurangan yang dapat diperbaiki, diantaranya sebagai berikut.

\begin{enumerate}
    \item Terdapat keperluan untuk \textit{error and trial} untuk memenuhi standar \textit{quality of service} yang berbentuk \textit{throughput} karena yang ditentukan \textit{autoscale} pada umumnya adalah permintaan serta limit pada prosesor dan memori, tidak memiliki data sebab akibat dari prosesor dan memori pada \textit{throughput}.
    \item Metrik yang digunakan untuk melakukan \textit{autoscale} hanya berdasarkan metrik umum sistem seperti utilisasi prosesor, memori serta beban permintaan dalam satuan waktu. Hal ini menyebabkan \textit{autoscale} tidak dapat memprediksi kebutuhan \textit{resource} pada waktu yang akan datang melalui pola penggunaan dan karakteristik \textit{container} yang ada didalamnya. Sehingga, efisiensi dari \textit{autoscale} tersebut sifatnya lebih "berjaga-jaga" atau \textit{just in case} dibanding mempersiapkan dari pengalaman sebelumnya. Masalah ini lebih timbul pada aplikasi atau sistem yang lebih general pada \textit{information retrieval} seperti \textit{Elasticseach} karena utilisasi prosesor dan memori tidak dapat menggambarkan kebutuhan \textit{resource} secara spesifik, melainkan harus ditelusuri dari kegunaannya.
    \item \textit{Metrics} yang digunakan adalah data \textit{realtime} sehingga jika ada turbulensi mendadak atau rutin akan mengurangi efisiensi dan performa karena \textit{autoscale} tipe statis akan terus berjalan dan menganggap hal tersebut normal. Tidak ada kondisi spesifik yang bisa ditetapkan oleh pengguna untuk membatasi \textit{autoscale} karena pengguna hanya meletakkan angka \textit{treshold} untuk \textit{autoscaler} melakukan pekerjaannya.
    \item \textit{Rolling Update} yang terlalu sering mengharuskan kubernetes memiliki sumber daya minimal sejumlah dua kali lipat. Selain itu, hal ini juga memiliki overhead karena \textit{Elastic Search} perlu melakukan beberapa proses saat dinyalakan.
\end{enumerate}

Pada tugas akhir ini, akan dilakukan penelitian untuk melakukan pengembangan metode \textit{autoscale} yang berjalan diatas Kubernetes yang spesifik untuk mengontrol alokasi sumber daya \textit{Elastic Search}. Dengan melakukan pengembangan tersebut, diharapkan penelitian ini dapat meningkatkan efisiensi \textit{autoscale} pada \textit{Elastic Search} melalui kontrol fleksibel dengan model prediktif berbasis \textit{time series}.

% Namun, saat ini kontrol adaptif milik Kubernetes memiliki beberapa kekurangan. Salah satu kekurangan dari kontrol adaptif kubernetes adalah \textit{trigger autoscale} yang didasarkan oleh metriks umum seperti utilisasi memori dan prosesor berdasarkan waktu saat itu. Sedangkan, tidak semua \textit{container} dinyatakan memakai sumber daya komputasi secara efisien jika hanya memakai faktor tersebut sebab tidak semua \textit{container} bergantung hanya pada utilisasi prosesor dan memori. Terkadang ada beberapa fitur pada sebuah \textit{container} yang terus berjalan namun tidak terpakai atau disimpan pada \textit{cache} namun tidak dipakai. Semuanya kembali lagi kepada isi dari sebuah \textit{container} serta penggunaannya. Selain dari hal tersebut, kubernetes juga hanya memakai data \textit{metrics} yang sedang berlangsung. Sehingga, kubernetes tidak dapat memprediksi kebutuhan \textit{resource} pada waktu yang akan datang melalui pola penggunaan. Hal ini akan menjadi acuan untuk meningkatkan efisiensi \textit{autoscale} dari kubernetes terutama pada \textit{Elastic Search} melalui adaptif kontrol dengan model prediktif berbasis \textit{time series}.