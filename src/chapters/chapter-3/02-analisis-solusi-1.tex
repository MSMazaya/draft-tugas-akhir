\section{Analisis Solusi}

Untuk melakukan penelitian pengembangan metode \textit{autoscale} tersebut, dilakukan pemetaan masalah, penanganan serta kebutuhan untuk melakukan penanganan tersebut.

\subsection{Pemetaan Masalah dan Penanganan}
\label{sec:pemetaan-masalah}
Masalah-masalah yang ada akan dipetakan dengan penanganan yang akan dilakukan. Pemetaan masalah dan penanganan dapat dilihat pada Tabel \ref{tab:pemetaan-masalah}.

\begin{table}[h]
    \caption{Tabel Pemetaan Masalah dan Penanganan}
    \vspace{0.25cm}
    \begin{center}
        \begin{tabular}{|c|p{2.5in}|p{2.5in}|}
            \rowcolor{gray!30}
            \hline
            \textbf{Nomor} & \textbf{Masalah} & \textbf{Penanganan} \tabularnewline
            \hline
            1 & Terdapat keperluan untuk \textit{error and trial} untuk memenuhi standar \textit{quality of service} yang berbentuk \textit{throughput} karena tidak memiliki data sebab akibat dari prosesor dan memori pada \textit{throughput}. &
            Menambah variabel \textit{throughput} untuk melakukan kontrol fleksibel.
            \tabularnewline

            2 &
            Kubernetes tidak bisa melakukan \textit{autoscale} berdasarkan variabel spesifik pada suatu kontainer, melainkan hanya generalisasi proses pada umumnya. &
            Mengembangkan \textit{autoscaler} pada ranah \textit{information retrieval} yang spesifik pada \textit{Elastic Search}. \tabularnewline

            3 &
            \textit{Metrics} yang digunakan adalah data \textit{realtime} sehingga tidak efisien terhadap data fluktuatif. &
            Mengembangkan \textit{autoscaler} dengan model prediksi berbasis \textit{time series} sehingga dapat melihat pola dari data historis.\tabularnewline

            4 & \textit{Quality of service} dan toleransi kepada \textit{tradeoff} antara efisiensi dan \textit{cost} dapat berbeda antara pengguna. &
            \textit{Autoscaler} harus fleksibel sehingga terdapat ruang untuk pengguna membuat aturan yang berjalan sesuai kondisi-kondisi yang disesuaikan dengan kebutuhan pemakai.\tabularnewline

            5 & \textit{Rolling Update} yang terlalu sering mengharuskan kubernetes memiliki sumber daya minimal sejumlah dua kali lipat. &
            Menggunakan teknologi \textit{In-Place Update of Pod Resources} \tabularnewline
            \hline
        \end{tabular}
    \end{center}
    \label{tab:pemetaan-masalah}
\end{table}