\section{Analisis Solusi}

Untuk melakukan penelitian pengembangan metode \textit{autoscale} tersebut, dilakukan pemetaan masalah, penanganan serta kebutuhan untuk melakukan penanganan tersebut.

\subsection{Pemetaan Masalah dan Penanganan}
\label{sec:pemetaan-masalah}
Masalah-masalah yang ada akan dipetakan dengan penanganan yang akan dilakukan. Pemetaan masalah dan penanganan dapat dilihat pada Tabel \ref{tab:pemetaan-masalah}.

\begin{table}[h]
    \caption{Tabel Pemetaan Masalah dan Penanganan}
    \vspace{0.25cm}
    \begin{center}
        \begin{tabular}{|c|p{2.5in}|p{2.5in}|}
            \hline
            Nomor & Masalah & Penanganan \tabularnewline
            \hline
            1 &
            Kubernetes tidak bisa melakukan \textit{autoscale} berdasarkan variabel spesifik pada suatu kontainer. &
            Kontrol fleksibel harus spesifik untuk suatu jenis kontainer, pada tugas akhir ini, akan berfokus pada \textit{Elastic Search}. \tabularnewline
            2 &
            \textit{Metrics} yang digunakan adalah data \textit{realtime} sehingga jika ada turbulensi akan mengurangi efisiensi dan performa. &
            Kontrol fleksibel memakai model prediksi berbasis \textit{time series} sehingga dapat melihat korelasi dengan data yang ada di masa lalu.\tabularnewline
            3 & Terdapat keperluan untuk \textit{error and trial} untuk memenuhi standar \textit{quality of service} yang berbentuk \textit{throughput} karena tidak memiliki data sebab akibat dari prosesor dan memori pada \textit{throughput}. &
            Menambah variabel \textit{throughput} untuk melakukan kontrol fleksibel.
            \tabularnewline
            4 & \textit{Rolling Update} yang terlalu sering mengharuskan kubernetes memiliki sumber daya minimal sejumlah dua kali lipat. &
            Menggunakan \textit{In-Place Update of Pod Resources} \tabularnewline
            5 & \textit{Quality of service} dan toleransi kepada \textit{tradeoff} antara efisiensi dan \textit{cost} dapat berbeda antara pengguna. &
            \textit{Autoscaler} harus memiliki ruang untuk pengguna membuat aturan yang berjalan sesuai kondisi-kondisi yang telah ditetapkan pengguna.\tabularnewline
            \hline
        \end{tabular}
    \end{center}
    \label{tab:pemetaan-masalah}
\end{table}