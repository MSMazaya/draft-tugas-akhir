\subsection{Rencana Penanganan Masalah dan Analisis Kebutuhan}

Masalah sudah dipetakan pada bagian \ref{sec:pemetaan-masalah} dan penanganan secara singkat telah dipaparkan. Bagian ini akan menguraikan kebutuhan secara detil untuk melakukan penanganan masalah tersebut.

\begin{enumerate}
    \item \textbf{(M1,M2) Menambah variabel \textit{throughput} dan spesifik terhadap\textit{Elastic Search}}
    
    Kebutuhan ini secara umum bisa didapatkan melalui \href{https://www.elastic.co/guide/en/elasticsearch/reference/current/cluster-nodes-stats.html}{\textit{Node Stats API}}. Untuk variabel yang akan ditarik, akan digunakan \textit{metrics response time} dari setiap operasi yang ada di \textit{Elastic Search} seperti \textit{Index, Get, Query, Fetch, Bulk, Refresh, Scroll, Suggest} dan \textit{Flush} serta metrik umum seperti utilisasi prosesor, beban tulis (\textit{write load}), serta memori.

    \item \textbf{(M3) Membuat model prediksi berbasis \textit{time series}}
    
    Kebutuhan ini bisa dilakukan dengan beberapa pendekatan seperti ARIMA dan LSTM. Kedua model ini sangat bagus dalam melakukan prediksi data yang dekat dengan korelasi waktu. Dalam tugas akhir ini, akan digunakan ARIMA untuk melakukan prediksi \textit{metrics} yang ada. Alasan menggunakan ARIMA adalah karena ARIMA merupakan model yang sederhana dan mudah untuk diimplementasikan, tidak perlu banyak riset dan percobaan terkait algoritma dan konfigurasi model. Selain itu, ARIMA juga merupakan model yang cukup baik untuk melakukan prediksi \textit{time series} yang memiliki \textit{trend} dan \textit{seasonality}.

    Sistem dapat secara preventif melihat prediksi di beberapa waktu yang akan datang, apabila naik dan tidak turun, ataupun turun dan tidak naik untuk waktu yang cukup lama, maka dapat dilakukan \textit{scaling}. Sehingga, dapat menghindari \textit{scaling} untuk waktu yang sangat singkat.

    \item \textbf{(M1,M4) Menerapkan \textit{rule-based} system yang ditetapkan oleh pengguna}
    
    Kebutuhan ini bisa dipenuhi dengan membuat bahasa \textit{scripting} yang sederhana untuk mengekspresikan \textit{rule}/kondisi yang ingin ditetapkan pengguna. Prediksi dan waktu prediksi suatu variabel dapat ditentukan oleh pengguna melalui bahasa \textit{scripting} sehingga pengguna dapat menggunakan model prediksi dengan leluasa. Selain itu, terdapat juga beberapa konfigurasi sistem kontrol fleksibel yang dapat dikonfigurasi oleh pengguna agar dapat menyesuaikan \textit{autoscaling} dengan kebutuhan.

    % \item \textbf{(M5) Menggunakan \textit{In-Place Update of Pod Resources}}

    % \textit{In-Place Update of Pod Resources} merupakan fitur baru yang dikembangkan oleh Kubernetes yang bertujuan untuk dapat mengubah ukuran pod secara dinamis tanpa melakukan \textit{restart}. Fitur ini baru saja diimplementasikan dan dipublikasikan dengan status \textit{alpha testing} sejak 12 Mei 2023, \parencite{kubeinplaceupdate}. Fitur ini berguna untuk melakukan \textit{resizing} tanpa melakukan \textit{restart} yang dapat digunakan melalui perintah \textit{kubectl} maupun \textit{Kubernetes Client Library}. Sistem yang akan diimplementasikan tidak akan memakai fitur ini dikarenakan masih \textit{alpha testing} dan masih baru. Eksperimen percobaan untuk menggunakan fitur ini bisa dilihat pada lampiran \ref{appendix:eksperimen-in-place-resource-resize}.

\end{enumerate}