\section{Analisis Solusi}

Untuk melakukan penelitian pengembangan kontrol adaptif tersebut, dilakukan pemetaan masalah, penanganan serta kebutuhan untuk melakukan penanganan tersebut.

\subsection{Pemetaan Masalah dan Penanganan}
\label{sec:pemetaan-masalah}
Masalah-masalah yang ada akan dipetakan dengan penanganan yang akan dilakukan. Pemetaan masalah dan penanganan dapat dilihat pada Tabel \ref{tab:pemetaan-masalah}.

\begin{table}[h]
    \caption{Tabel Pemetaan Masalah dan Penanganan}
    \vspace{0.25cm}
    \begin{center}
        \begin{tabular}{|c|p{2.5in}|p{2.5in}|}
            \hline
            Nomor & Masalah & Penanganan \tabularnewline
            \hline
            1 &
            Kubernetes tidak bisa melakukan kontrol adaptif berdasarkan variabel spesifik pada suatu kontainer. &
            Kontrol adaptif harus spesifik untuk suatu jenis kontainer, pada tugas akhir ini, akan berfokus pada \textit{Elastic Search}. \tabularnewline
            2 &
            \textit{Metrics} yang digunakan adalah data \textit{realtime} sehingga jika ada turbulensi akan mengurangi efisiensi dan performa. &
            Kontrol adaptif memakai model prediksi berbasis \textit{time series} sehingga dapat melihat korelasi dengan data yang ada di masa lalu.\tabularnewline
            3 & Terdapat keperluan untuk \textit{error and trial} untuk memenuhi standar \textit{quality of service} yang berbentuk \textit{throughput} karena tidak memiliki data sebab akibat dari prosesor dan memori pada \textit{throughput}. &
            Menambah variabel \textit{throughput} untuk melakukan kontrol adaptif.
            \tabularnewline
            4 & \textit{Rolling Update} yang terlalu sering mengharuskan kubernetes memiliki sumber daya minimal sejumlah dua kali lipat. &
            Menggunakan \textit{In-Place Update of Pod Resources} \tabularnewline
            5 & \textit{Quality of service} dan toleransi kepada \textit{tradeoff} antara efisiensi dan \textit{cost} dapat berbeda antara pengguna. &
            Kontrol adaptif harus memiliki ruang untuk pengguna membuat aturan terhadap kontrol adaptif.\tabularnewline
            \hline
        \end{tabular}
    \end{center}
    \label{tab:pemetaan-masalah}
\end{table}

\subsection{Rencana Penanganan Masalah dan Analisis Kebutuhan}

Masalah sudah dipetakan pada bagian \ref{sec:pemetaan-masalah} dan penanganan secara kasar telah dirancangkan. Dalam melakukan penanganan masalah tersebut, diperlukan beberapa kebutuhan yang akan dijelaskan pada bagian ini.

\begin{enumerate}
    \item \textbf{Menambah variabel untuk melakukan kontrol adaptif yang spesifik pada \textit{throughput} dan spesifikasi \textit{Elastic Search}}
    
    Kebutuhan ini secara umum bisa didapatkan melalui \href{https://www.elastic.co/guide/en/elasticsearch/reference/current/cluster-nodes-stats.html}{\textit{Node Stats API}}. Untuk variabel yang akan ditarik, akan digunakan \textit{metrics response time} dari setiap operasi yang ada di \textit{Elastic Search} seperti \textit{Index, Get, Query, Fetch, Bulk, Refresh, Scroll, Suggest} dan \textit{Flush} serta \textit{metrics} umum seperti utilisasi prosesor serta memori.

    \item \textbf{Membuat model prediksi berbasis \textit{time series}}
    
    Kebutuhan ini bisa dilakukan dengan beberapa pendekatan seperti ARIMA, SARIMA, LSTM serta variasi seperti ARIMA. Dalam tugas akhir ini, akan digunakan ARIMA untuk melakukan prediksi \textit{metrics} yang ada. Alasan menggunakan ARIMA adalah karena ARIMA merupakan model yang sederhana dan mudah untuk diimplementasikan. Selain itu, ARIMA juga merupakan model yang cukup baik untuk melakukan prediksi \textit{time series} yang memiliki \textit{trend} dan \textit{seasonality}. Ditambah, variasi dari ARIMA memiliki masalah ketika data memiliki standar deviasi rendah, contohnya adalah VARMAX atau \textit{Vector Auto Regressive Moving Average with Exogenous Regressors}. Meskipun dengan satu model VARMAX, koneksi antara variabel juga bisa ditentukan, namun ARIMA juga sudah cukup baik untuk memprediksi variabel-variabel secara terpisah.

    \item \textbf{Menggunakan \textit{In-Place Update of Pod Resources}}

    \textit{In-Place Update of Pod Resources} merupakan fitur baru yang dikembangkan oleh Kubernetes yang bertujuan untuk dapat mengubah ukuran pod secara dinamis tanpa melakukan \textit{restart}. Fitur ini baru saja dicoba diimplementasikan dan dipublikasikan dengan status \textit{alpha testing} sejak 12 Mei 2023, \parencite{kubeinplaceupdate}. Fitur ini layak diteliti untuk melakukan \textit{resizing} tanpa melakukan \textit{restart}. Sebagai catatan, saat tugas akhir ini dikerjakan, fitur ini masih memiliki banyak keterbatasan dan masalah yang tertulis pada blog kubernetes maupun belum tertulis.

    \item \textbf{Menerapkan \textit{rule} untuk kontrol adaptif yang ditetapkan pengguna}
    
    Kebutuhan ini bisa dipenuhi dengan membuat bahasa \textit{scripting} yang simpel untuk mengekspresikan \textit{rule}/kondisi yang ingin ditetapkan pengguna. Prediksi dan waktu prediksi suatu variabel dapat ditentukan oleh pengguna melalui bahasa \textit{scripting} sehingga pengguna dapat menggunakan model prediksi dengan leluasa. Selain itu, terdapat juga beberapa konfigurasi sistem yang dapat dikonfigurasi oleh pengguna agar dapat menyesuaikan sistem kontrol adaptif dengan kebutuhan.

\end{enumerate}